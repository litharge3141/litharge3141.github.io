\documentclass[dvipdfmx,autodetect-engine]{jsarticle}


\usepackage[utf8]{inputenc}
\usepackage{amsmath}
\usepackage{amsthm}
\usepackage{amssymb}
\usepackage{mathtools}

\usepackage[dvipdfmx]{graphicx}

\newtheorem{theorem}{Theorem}[section]
\newtheorem{corollary}{Corollary}[theorem]
\newtheorem{lemma}[theorem]{Lemma}

\theoremstyle{remark}
\newtheorem*{remark}{Remark}

\theoremstyle{definition}
\newtheorem{definition}{Definition}[section]

\renewcommand{\labelenumi}{(\arabic{enumi})}
\renewcommand{\labelenumii}{(\alph{enumii}}
\newcommand{\R}{\mathbb{R}}
\newcommand{\N}{\mathbb{N}}
\newcommand{\C}{\mathbb{C}}


\begin{document}

\title{Constantin Lax Majda方程式}
\author{@litharge3141}
\date{\today}
\maketitle

\begin{abstract}
    3次元Euler方程式の時間大域的な弱解の存在は知られておらず,常に有限時間で無限大に
    爆発すると予想している人も多い.このノートでは,そのような爆発のメカニズムを
    調べるためのモデル方程式として知られているConstantin-Lax-Majda方程式について
    解説する.
\end{abstract}

\section{導出}
3次元Euler方程式
\begin{align*}
   \frac{\partial u}{\partial t} + u \cdot \nabla u &= - \nabla p \\
   \mathrm{div}u &=0
\end{align*}
を考える.$\nabla p$は厄介なので$\mathrm{rot}$を施して消去する.
両辺に$\mathrm{rot}$を施して十分な滑らかさを仮定すると,$\omega \coloneqq \mathrm{rot} u$として
\begin{align*}
    \frac{\partial \omega}{\partial t} + u \cdot \nabla \omega &= \omega \cdot \nabla u \\
    \mathrm{div}\omega &=0
\end{align*}
という$\omega$と$u$の方程式を得る.
\begin{align*}
    \omega &= \mathrm{rot} u \\
    \mathrm{div} u &= 0
\end{align*}
から$\mathrm{rot} \omega = - \triangle u$を得る.これはPoisson方程式だから3次元の基本解と
畳み込んでBiot-Savart積分
\begin{align*}
    u(x) = -\frac{1}{4\pi} \int_{\R^3} \frac{(x-y)\times \omega(y)}{{\lvert x-y \rvert}^3} \mathrm{d}x
\end{align*}
を得る.これを上述の$\omega$と$u$の方程式に代入したものを渦度方程式という.
2次元でも同様の操作をすることはできるが,
3次元との最大の違いは渦の伸長項と呼ばれる$\nabla u \cdot \omega$の有無で,
この項のために3次元では大域解の存在が証明できていないと考えられる.
これらを上手く1次元の方程式で真似をして,モデルを構成する.
\section{厳密解の構成}
\section{拡張やそのほかの性質}
    


\end{document}