\documentclass[dvipdfmx,autodetect-engine]{jsarticle}


\usepackage[utf8]{inputenc}
\usepackage{amsmath}
\usepackage{amsthm}
\usepackage{amssymb}
\usepackage{mathtools}

\usepackage[dvipdfmx]{graphicx}

\newtheorem{theorem}{Theorem}[section]
\newtheorem{corollary}{Corollary}[theorem]
\newtheorem{lemma}[theorem]{Lemma}

\theoremstyle{remark}
\newtheorem*{remark}{Remark}

\theoremstyle{definition}
\newtheorem{definition}{Definition}[section]

\renewcommand{\labelenumi}{(\arabic{enumi})}
\renewcommand{\labelenumii}{(\alph{enumii}}
\newcommand{\R}{\mathbb{R}}
\newcommand{\N}{\mathbb{N}}
\newcommand{\C}{\mathbb{C}}
\newcommand{\diver}{\mathrm{div} \,}
\newcommand{\rot}{\mathrm{rot} \,}


\begin{document}

\title{Constantin Lax Majda方程式}
\author{@litharge3141}
\date{\today}
\maketitle

\begin{abstract}
    3次元Euler方程式の時間大域的な弱解の存在は知られておらず,常に有限時間で無限大に
    爆発すると予想している人も多い.このノートでは,そのような爆発のメカニズムを
    調べるためのモデル方程式として知られているConstantin-Lax-Majda方程式について
    解説する.
\end{abstract}

\section{導出}
\subsection{3次元渦度方程式}
Constantin-Lax-Majda方程式(CLMと略す)は3次元渦度方程式の構造を
1次元で真似をして少しでも理解を前に進めようというものだから,まず
3次元渦度方程式の導出をする.3次元Euler方程式
\begin{align*}
   \frac{\partial u}{\partial t} + u \cdot \nabla u &= - \nabla p \\
   \diver u &=0
\end{align*}
を考える.$\nabla p$は厄介なので$\rot$を施して消去する.
両辺に$\rot$を施して十分な滑らかさを仮定すると,$\omega \coloneqq \rot u$として
\begin{align*}
    \frac{\partial \omega}{\partial t} + u \cdot \nabla \omega &= \omega \cdot \nabla u \\
    \diver\omega &=0
\end{align*}
という$\omega$と$u$の方程式を得る.ここで
\begin{align*}
    \omega &= \rot u \\
    \diver u &= 0
\end{align*}
から$\rot \omega = - \triangle u$を得るが,これはPoisson方程式だから3次元の基本解と
畳み込んで部分積分するとBiot-Savart積分
\begin{align*}
    u(t,x) = -\frac{1}{4\pi} \int_{\R^3} \frac{(x-y)\times \omega(t,y)}{{\lvert x-y \rvert}^3} \mathrm{d}x
\end{align*}
が得られる.これを
\begin{align*}
    \frac{\partial \omega}{\partial t} + u \cdot \nabla \omega = \omega \cdot \nabla u 
\end{align*}
と連立したものを渦度方程式という.
2次元でも同様の操作をすることはできるが,
3次元との最大の違いは渦の伸長項と呼ばれる$\nabla u \cdot \omega$の有無で,
この項のために3次元では大域解の存在が証明できていないと考えられている.

\subsection{CLMの導出}
Biot-Savart積分から$\nabla u$を求めようとすると,
Poisson方程式の基本解の微分と$\omega$の畳み込みが現れる.
これは特異積分作用素と呼ばれるクラスに入る作用素である.
空間1次元ではこのような作用素と同じ性質を持つものとして
ヒルベルト変換$H$が知られているから,これを用いる.
すなわち,$u_x = H(\omega)$としてBiot-Savart積分(の微分)
を模することにする.爆発解の存在に移流項$u\cdot \nabla \omega$
は関わらないと考えて,簡単のためこの項を0にする.
すると1次元Constantin-Lax-Majda方程式
\begin{align*}
    \omega_t = H(\omega)\omega 
\end{align*}
が得られる.
\section{厳密解の構成}
\subsection{主定理}
CLMの$\R$における初期値問題
\begin{align*}
    \omega_t &= H(\omega)\omega \\
    \omega(0,x) &= \omega_0(x)
\end{align*}
を考える.ここで,ヒルベルト変換$H:H^1(\R)\to H^1(\R)$は
\begin{align*}
    H(\omega) \coloneqq \frac{1}{\pi} \mathrm{p.v.} \int_{\R} \frac{\omega(y)}{x-y} \mathrm{d}y
\end{align*}
によって与えられる.
この時,次が成り立つ.
\begin{theorem}\label{main_theorem}
    $\omega_0 \in C^{\infty}(\R) \cap H^{1}(\R)$の時,CLMの初期値問題の解は
    \begin{align*}
        \omega (t,x) = \frac{4\omega_0 (x)}{(2-tH(\omega_0)(x))^2 + t^2 \omega_0^2(x)}
    \end{align*}
    によって与えられる.
\end{theorem}

\subsection{補題や用いる性質}
$\mathrm{Theorem} \ref{main_theorem}$の証明中に用いる$H$の性質などをすべて証明する.
\section{拡張やそのほかの性質}
    


\end{document}