\documentclass[dvipdfmx,autodetect-engine]{jsarticle}


\usepackage[utf8]{inputenc}
\usepackage{amsmath}
\usepackage{amsthm}
\usepackage{amssymb}
\usepackage{mathtools}

\usepackage[dvipdfmx]{graphicx}

\newtheorem{theorem}{Theorem}[section]
\newtheorem{corollary}{Corollary}[theorem]
\newtheorem{lemma}[theorem]{Lemma}
\newtheorem{example}{Example}[section]

\theoremstyle{remark}
\newtheorem*{remark}{Remark}

\theoremstyle{definition}
\newtheorem{definition}{Definition}[section]
\mathtoolsset{showonlyrefs}

\renewcommand{\labelenumi}{(\arabic{enumi})}
\renewcommand{\labelenumii}{(\alph{enumii}}
\newcommand{\R}{\mathbb{R}}
\newcommand{\N}{\mathbb{N}}
\newcommand{\C}{\mathbb{C}}
\newcommand{\Z}{\mathbb{Z}}
\newcommand{\diver}{\mathrm{div} \,}
\newcommand{\rot}{\mathrm{rot} \,}
\newcommand{\sgn}{\mathrm{sgn} \,}


\begin{document}

\title{Constantin-Lax-Majda方程式}
\author{@litharge3141}
\date{\today}
\maketitle

\begin{abstract}
    3次元Euler方程式の時間大域的な弱解の存在は知られておらず,常に有限時間で無限大に
    爆発すると予想している人も多い.このノートでは,そのような爆発のメカニズムを
    調べるためのモデル方程式として知られているConstantin-Lax-Majda方程式について
    元論文に基づいて解説する.
\end{abstract}

\section{導出}
\subsection{3次元渦度方程式}
Constantin-Lax-Majda方程式(CLMと略す)は3次元渦度方程式の構造を
1次元で真似をして少しでも理解を前に進めようというものだから,まず
3次元渦度方程式の導出をする.3次元Euler方程式
\begin{align}
    \begin{dcases}
        \frac{\partial u}{\partial t} + u \cdot \nabla u = - \nabla p \\
        \diver u =0
    \end{dcases}
\end{align}
を考える.$\nabla p$は厄介なので$\rot$を施して消去する.
両辺に$\rot$を施して十分な滑らかさを仮定すると,$\omega \coloneqq \rot u$として
\begin{align}
    \begin{dcases}
        \frac{\partial \omega}{\partial t} + u \cdot \nabla \omega = \omega \cdot \nabla u \label{vor1} \\
        \diver\omega =0
    \end{dcases}
\end{align}
という$\omega$と$u$の方程式を得る.ここで
\begin{align}
    \begin{dcases}
        \omega = \rot u \\
        \diver u = 0
    \end{dcases}
\end{align}
から$\rot \omega = - \triangle u$を得るが,これはPoisson方程式だから3次元の基本解と
畳み込んで部分積分するとBiot-Savart積分
\begin{align}
    u(t,x) = -\frac{1}{4\pi} \int_{\R^3} \frac{(x-y)\times \omega(t,y)}{{\lvert x-y \rvert}^3} \mathrm{d}x
\end{align}
が得られる.これを式$\eqref{vor1}$と連立したものを渦度方程式という.
2次元でも同様の操作をすることはできるが,
3次元との最大の違いは渦の伸長項と呼ばれる$\nabla u \cdot \omega$の有無で,
この項のために3次元では大域解の存在が証明できていないと考えられている.

\subsection{CLMの導出}
Biot-Savart積分から$\nabla u$を求めようとすると,
Poisson方程式の基本解の微分と$\omega$の畳み込みが現れる.
これは特異積分作用素と呼ばれるクラスに入る作用素である.
空間1次元ではこのような作用素と同じ性質を持つものとして
ヒルベルト変換$H$が知られているから,これを用いる.
すなわち,$u_x = H(\omega)$としてBiot-Savart積分(の微分)
を模することにする.爆発解の存在に移流項$u\cdot \nabla \omega$
は関わらないと考えて,簡単のためこの項を0にする.
すると1次元Constantin-Lax-Majda方程式
\begin{align}
    \omega_t = H(\omega)\omega 
\end{align}
が得られる.
\section{厳密解の構成}
\subsection{主定理}
CLMの$\R$における初期値問題
\begin{align}
    \begin{dcases}
        \omega_t = H(\omega)\omega \\
        \omega(0,x) = \omega_0(x)
    \end{dcases}
\end{align}
を考える.ここで,ヒルベルト変換$H:H^1(\R)\to H^1(\R)$は
\begin{align}
    H(\omega) \coloneqq \frac{1}{\pi} \mathrm{p.v.} \int_{\R} \frac{\omega(y)}{x-y} \mathrm{d}y
\end{align}
によって与えられる.
この時,次が成り立つ.
\begin{theorem}\label{main_theorem}
    $\omega_0 \in C^{\infty}(\R) \cap H^{1}(\R)$の時,CLMの初期値問題の解は
    \begin{align}
        \omega (t,x) = \frac{4\omega_0 (x)}{(2-tH(\omega_0)(x))^2 + t^2 \omega_0^2(x)}
    \end{align}
    によって与えられる.
\end{theorem}

\subsection{補題や用いる性質}
$\mathrm{Theorem} \ref{main_theorem}$の証明中に用いる$H$の性質などをすべて証明する.
$L^2$におけるフーリエ変換の知識は仮定する.
特に$H$の合成と積に対する振る舞いが重要である.$H^1$有界性から始める.
\begin{theorem}[$L^2$有界性]
    $\delta >0$が与えられたとする.$H_{\delta}$を$f \in L^2 (\R)$に対して
    \begin{align}
        H_{\delta} (f) (x) \coloneqq \frac{1}{\pi} \int_{\lvert x-y \rvert > \delta} 
        \frac{f(y)}{x-y} \mathrm{d}y
    \end{align}
    によって定義すると,$H_{\delta}(f) \in L^2(\R)$であり,
    $\delta \to 0$で$H_{\delta}(f)$は$L^2$収束する.
\end{theorem}

\begin{proof}
    フーリエ変換を利用して証明をする.
    \begin{align}
        h_{\delta}(x) = 
        \begin{dcases}
            0\quad (\lvert x \rvert < \delta) \\
            \frac{1}{x} \quad (\lvert x \rvert \geq \delta)
        \end{dcases}
    \end{align}
    によって$h_\delta$を定める.$\mathcal{F}(h_\delta)$を複素積分を用いて計算すると,
    $\mathcal{F}(h_\delta)$は$\delta$によらずに有界であることが分かる.さらに,
    $\lim_{\delta \to 0} \mathcal{F}(h_\delta) (\xi) = -i \sqrt{\pi / 2} \, \sgn \xi$
    が成立する.よって畳み込みの性質から
    \begin{align}
        \mathcal{F}(H_\delta (f)) &= \mathcal{F} (h_\delta \ast f) \\
            &= \mathcal{F}(h_\delta) \mathcal{F}(f)
    \end{align}
    が成立し,ルベーグの収束定理から$\mathcal{F}(H_\delta (f))$は$\delta\to 0$で
    $-i \sqrt{\pi / 2} \, \sgn \xi \, \mathcal{F}(f)$に$L^2$収束する.
    フーリエ変換の$L^2$等長性より,$H_\delta (f)$も$L^2$収束する.
\end{proof}

定理の証明から$H(f)$のフーリエ変換は$-i \sqrt{\pi / 2} \, \sgn \xi \, \mathcal{F}(f)$
で与えられる.これはフーリエ掛け算作用素と呼ばれる形をしており,この表示は今後も用いる.
\section{拡張やそのほかの性質}
    


\end{document}