\documentclass[dvipdfmx,autodetect-engine]{jsarticle}


\usepackage[utf8]{inputenc}
\usepackage{amsmath}
\usepackage{amsthm}
\usepackage{amssymb}
\usepackage{mathtools}

\usepackage[dvipdfmx]{graphicx}

\newtheorem{theorem}{Theorem}[section]
\newtheorem{corollary}{Corollary}[theorem]
\newtheorem{lemma}[theorem]{Lemma}

\theoremstyle{remark}
\newtheorem*{remark}{Remark}

\theoremstyle{definition}
\newtheorem{definition}{Definition}[section]

\renewcommand{\labelenumi}{(\arabic{enumi})}
\renewcommand{\labelenumii}{(\alph{enumii}}
\newcommand{\R}{\mathbb{R}}
\newcommand{\N}{\mathbb{N}}
\newcommand{\C}{\mathbb{C}}


\begin{document}

\title{マルコフ連鎖の基本}
\author{@litharge3141}
\date{\today}
\maketitle

\section{マルコフ連鎖の基本}
有限ないし高々可算の状態空間を持ち,かつ離散的という最もシンプルな場合を通して,
マルコフ過程の概念を整理する.大数の強法則を証明したことがある程度の知識を仮定する.
本文全体を通して$I$を高々可算集合とし,その$\sigma$-代数として
$I$の部分集合全体$\mathcal{P}(I)$を取る.また,自然数の全体$\N$は$0$を含むものとする.
\subsection{マルコフ連鎖の定義}
時間変化する確率変数は,各時刻で$I$上の確率分布を与える.
それが時間無限大でどうなるかとか,そういう問題を考えたい.
$I$は高々可算だから,$I$上の分布はうまく言い換えられる.


\begin{definition}
    写像$\nu : I \to [0,1]$が確率ベクトルであるとは,$\sum_{i\in I} \nu (i)=1$
    が成立することをいう.$\nu$を$(\nu_i)_{i\in I}$ともかき,$\nu(i)$を単に$\nu_i$とかく.
    写像$A:I\times I \to [0,1]$が確率行列であるとは,任意の$i\in I$に対して,
    $\sum_{j\in I} A(i,j) = 1$が成立することをいう.$A$を$(A_{ij})_{i,j\in I}$ともかき,
    $A(i,j)$を単に$A_{ij}$とかく.
\end{definition}


確率ベクトル$\nu$に対して$P:\mathcal{P}(I) \to [0,1]$を$P(E) \coloneqq \sum_{i \in E} \nu_i$により
定めると,$P$は$I$上の確率測度となる.逆に$I$上の確率測度$P$に対して
$\nu_i \coloneqq P(\{i\})$と定めると,$\nu$は確率ベクトルになる.
したがって,$I$上の確率分布を定めることは,確率ベクトルを定めることと同じである.


\begin{theorem}
    確率ベクトル$\nu$と確率行列$A$の積$\nu A : I \to [0,1]$を$j \in I$に対し
    $\nu A (j) \coloneqq \sum_{i \in I} \nu_i A_{ij}$によって定めることができ,$\nu A$は
    確率ベクトルになる.
    確率行列$A,B$の積$AB :I\times I \to [0,1]$を$AB(i,j) = \sum_{k \in I} A_{ik}B_{kj}$
    によって定めることができ,$AB$は確率行列になる.
\end{theorem}


\begin{proof}
    非負項の二重級数はいつでも和をとる順序を交換できるので,定理が従う.
\end{proof}


確率行列は確率ベクトルの変換を定めるが,これは分布の変換を定めているのと
同じことである.これをもとにしてマルコフ連鎖の定義を与える.


\begin{definition}
    確率ベクトル$\nu$と確率行列$A$が与えられたとする.
    $(\Omega,\mathcal{F},P)$を確率空間として,
    $\Omega$から$I$への確率変数列$(X_n)_{n=0}^{\infty}$が
    遷移行列$A$,初期分布$\nu$をもつマルコフ連鎖であるとは,
    任意の$n \in \N$と任意の$i_0,\ldots,i_{n} \in I$に対して,
    $P(X_{0} = i_{0}, X_{1}=i_{1},\ldots,X_n = i_n) = 
    \nu_{i_0} A_{i_0 i_1} A_{i_1 i_2}\cdots A_{i_{n-1} i_n}$
    が成立することをいう.
\end{definition}


初期分布と時間発展による分布の変換が(時刻によらない一定の法則で)与えられているとき,
それにしたがって発展するような分布を持つ確率変数列のことをマルコフ連鎖という.
残念ながら(独立同分布列と同様に)常に存在するかは自明ではないので,先にそれを示す.


\begin{theorem}
    確率ベクトル$\nu$と確率行列$A$が与えられたとする.このとき,
    ある確率空間$(\Omega,\mathcal{F},P)$と
    $\Omega$から$I$への確率変数列$(X_n)_{n=0}^{\infty}$が
    存在して,遷移行列$A$,初期分布$\nu$をもつマルコフ連鎖となる.
\end{theorem}

\begin{proof}
    $\R$に埋め込んで拡張定理,追記予定
\end{proof}


\end{document}