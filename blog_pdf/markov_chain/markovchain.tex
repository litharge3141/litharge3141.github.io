\documentclass[dvipdfmx,autodetect-engine]{jsarticle}


\usepackage[utf8]{inputenc}
\usepackage{amsmath}
\usepackage{amsthm}
\usepackage{amssymb}
\usepackage{mathtools}

\usepackage[dvipdfmx]{graphicx}

\newtheorem{theorem}{Theorem}[section]
\newtheorem{corollary}{Corollary}[theorem]
\newtheorem{lemma}[theorem]{Lemma}

\theoremstyle{remark}
\newtheorem*{remark}{Remark}

\theoremstyle{definition}
\newtheorem{definition}{Definition}[section]

\renewcommand{\labelenumi}{(\arabic{enumi})}
\renewcommand{\labelenumii}{(\alph{enumii}}
\newcommand{\R}{\mathbb{R}}
\newcommand{\N}{\mathbb{N}}
\newcommand{\C}{\mathbb{C}}


\begin{document}

\title{マルコフ連鎖の基本}
\author{@litharge3141}
\date{\today}
\maketitle

\section{マルコフ連鎖の基本}
有限ないし高々可算の状態空間を持ち,かつ離散的という最もシンプルな場合を通して,
マルコフ過程の概念を整理する.大数の強法則を証明したことがある程度の知識を仮定する.
本文全体を通して$I$を高々可算集合とし,その$\sigma$-代数として$I$の部分集合全体を取る.
\subsection{マルコフ連鎖の定義}
サイコロをふるという試行の確率モデルを考える.どの本にも書いてあることだが,$(\Omega,\mathcal{F},P)$を確率空間として,
何らかの確率変数$X:\Omega \to \{1,2,3,4,5,6\}$で,$P(X=i)=1/6$が$1\leq i\leq 6$で成立するようなものを考えることになる.
$X$の定義域は実際には目に見えず,見えるのは出た目だけである.

\begin{definition}
    $(\Omega,\mathcal{F},P)$を確率空間とする.
\end{definition}
\end{document}