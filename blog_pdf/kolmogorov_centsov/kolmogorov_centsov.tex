\documentclass[dvipdfmx,autodetect-engine]{jsarticle}


\usepackage[utf8]{inputenc}
\usepackage{amsmath}
\usepackage{amsthm}
\usepackage{amssymb}
\usepackage{mathtools}

\usepackage[dvipdfmx]{graphicx}

\newtheorem{theorem}{Theorem}[section]
\newtheorem{corollary}{Corollary}[theorem]
\newtheorem{lemma}[theorem]{Lemma}
\newtheorem{example}{Example}[section]

\theoremstyle{remark}
\newtheorem*{remark}{Remark}

\theoremstyle{definition}
\newtheorem{definition}{Definition}[section]
\mathtoolsset{showonlyrefs}

\renewcommand{\labelenumi}{(\arabic{enumi})}
\renewcommand{\labelenumii}{(\alph{enumii}}
\newcommand{\R}{\mathbb{R}}
\newcommand{\N}{\mathbb{N}}
\newcommand{\C}{\mathbb{C}}
\newcommand{\Z}{\mathbb{Z}}
\newcommand{\diver}{\mathrm{div} \,}
\newcommand{\rot}{\mathrm{rot} \,}
\newcommand{\abs}[1]{\left\lvert#1\right\rvert}
\newcommand{\norm}[1]{\left\lVert#1\right\rVert}
\newcommand{\setmid}{\mathrel{} \middle| \mathrel{}}


\begin{document}

\title{Kolmogorov-Centsovの定理}
\author{@litharge3141}
\date{\today}
\maketitle

\begin{abstract}
    このノートでは,確率過程の$L^p$ノルム評価から
    ヘルダー連続性を導くKolmogorov-Centsovの定理を証明する.
    この定理はブラウン運動の構成や,確率微分方程式の解の評価など,
    種々の確率過程の連続性の証明に広く用いられる.
\end{abstract}

\section{定理の主張と証明}
\begin{theorem}[Kolmogorov-Centsovの定理]
    $(\Omega,\mathcal{F},P)$を確率空間とし,$(B,\norm{\cdot})$
    をBanach空間とそのノルムの組とする.
    $X=(X_{t})_{t \in [0,T]}$を$\Omega$から$B$への確率過程とする.
    \begin{align}
        \exists C,p,q>0,\quad \forall s,t \in [0,T],
        \quad E[\norm{X_{t} - X_{s}}^p] \leq C \abs{t-s}^{1+q}
    \end{align}
    が成立すると仮定する.このとき,任意の$0 < r < q/p$に対して,
    ある$\Omega$から$B$への確率過程$\tilde{X}=(\tilde{X}_{t})_{t \in [0,T]}$
    が存在して,次を満たす.
    \begin{itemize}
        \item 任意の$t \in [0,T]$に対して
        $X_t = \tilde{X}_t$がほとんどいたるところ成立する.
        \item ほとんどいたるところ正の値を取るある確率変数$h\colon \Omega \to \R$
        とある$M>0$が存在して
        \begin{align}
            P\left(\omega \setmid 
            \sup_{\substack{0 < \abs{t-s} < h(\omega) \\ s,t \in [0,T]}}
            \frac{\norm{\tilde{X}_{t}(\omega) - 
            \tilde{X}_{s}(\omega)}}{\abs{t-s}^{r}} 
            \leq M
            \right)=1
        \end{align}
        が成立する.
    \end{itemize}
    特に,$X$は$r$次H\"{o}lder連続な修正$\tilde{X}$を持つ.
\end{theorem}

\begin{proof}
    $0<r<q/p$が任意に与えられたとする.
    $D_{n} \coloneqq \left\{\frac{kT}{2^n} \setmid k=0,1,\ldots,2^{n}\right\}$とし,
    $D \coloneqq \bigcup_{n=1}^{\infty} D_n$とおく.$X_{t}$は$D$上一様連続な確率過程であることを示す.
    \begin{align}
        P\left(\omega \setmid 
        \exists n^{*}(\omega)\in \N,\,\forall n \geq n^{*}(\omega),\,
        \max_{1\leq k\leq 2^{n}} \norm{X_{\frac{k}{2^n}}(\omega) - X_{\frac{k-1}{2^n}}(\omega)}
        < 2^{-rn}\right) = 1
    \end{align}
    を示す.
    \begin{align}
        A_{n} \coloneqq \left\{ \omega \setmid 
        \max_{1\leq k\leq 2^n} \norm{X_{\frac{k}{2^n}}(\omega) - 
        X_{\frac{k-1}{2^n}}(\omega)} < 2^{-rn}\right\}
    \end{align}
    とおくと,$P(\liminf_{n\to\infty} A_{n})=1$を示せばよい.
    それは$P(\limsup_{n\to\infty}A_{n}^{c})=0$と同値である.
    さらに,Borel-Canteliの補題から$\sum_{n=1}^{\infty}P(A_{n}^{c})$が収束することを
    証明すれば十分である.これを示す.
    $s,t \in [0,T]$に対して,
    \begin{align}
        P(\norm{X_{t} - X_{s}} \geq \varepsilon) 
        &\leq \frac{1}{\varepsilon^{p}} E[\norm{X_{t} - X_{s}}^{p}] 
        \quad (\because \text{Chebyshevの不等式})\\
        &\leq  \frac{C\abs{t-s}^{1+q}}{\varepsilon^{p}} \quad (\because \text{仮定})
    \end{align}
    を得る.したがって$t=k/2^{n},s=(k-1)/2^{n},\varepsilon = 2^{-rn}$とすると
    \begin{align}
        P\left(\norm{ X_{\frac{k}{2^{n}}} - X_{\frac{k-1}{2^{n}}} } \geq 2^{-rn}\right)
        \leq C 2^{-n(1+q)}2^{rpn} = C 2^{-n(1+q-pr)}
    \end{align}
    となる.したがって
    \begin{align}
        P(A_{n}^{c}) &= P\left( \bigcup_{k=1}^{2^{n}} 
        \norm{ X_{\frac{k}{2^{n}}} - X_{\frac{k-1}{2^{n}}} } \geq 2^{-rn} \right)\\
        &\leq \sum_{k=1}^{2^{n}} P\left(
            \norm{ X_{\frac{k}{2^{n}}} - X_{\frac{k-1}{2^{n}}} } \geq 2^{-rn}
         \right)\\
         &\leq C2^{-n(q-pr)}
    \end{align}
    を得る.$q-pr>0$だから,$\sum_{n=1}^{\infty} P(A_{n}^{c}) < \infty$となる.
    したがって示された.
\end{proof}

\section{応用例}
\end{document}