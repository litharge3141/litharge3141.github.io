\documentclass[dvipdfmx,autodetect-engine]{jsarticle}


\usepackage[utf8]{inputenc}
\usepackage{amsmath}
\usepackage{amsthm}
\usepackage{amssymb}
\usepackage{mathtools}

\usepackage[dvipdfmx]{graphicx}

\newtheorem{theorem}{Theorem}[section]
\newtheorem{corollary}{Corollary}[theorem]
\newtheorem{lemma}[theorem]{Lemma}
\newtheorem{example}{Example}[section]

\theoremstyle{remark}
\newtheorem*{remark}{Remark}

\theoremstyle{definition}
\newtheorem{definition}{Definition}[section]
\mathtoolsset{showonlyrefs}

\renewcommand{\labelenumi}{(\arabic{enumi})}
\renewcommand{\labelenumii}{(\alph{enumii}}
\newcommand{\R}{\mathbb{R}}
\newcommand{\N}{\mathbb{N}}
\newcommand{\C}{\mathbb{C}}
\newcommand{\Z}{\mathbb{Z}}
\newcommand{\diver}{\mathrm{div} \,}
\newcommand{\rot}{\mathrm{rot} \,}
\newcommand{\abs}[1]{\left\lvert#1\right\rvert}
\newcommand{\norm}[1]{\left\lVert#1\right\rVert}


\begin{document}

\title{SDEの数値計算 強収束}
\author{@litharge3141}
\date{\today}
\maketitle

\abstract{}
SDEの数値計算の中で最も基本的なEuler-MaruyamaおよびMilsteinのスキームについて,
その厳密解への強収束と呼ばれる収束について述べる.
これは数値計算の各時間ステップについて,その厳密解からの分散が
刻み幅$h$を用いて上から評価できるというものである.

\section{}

\subsection{準備}
主定理を述べる前に必要な用語などについて述べる.
数値計算においては確率微分方程式の強解を近似して計算する.
\begin{definition}\label{strong_solution}
    $(\Omega,\mathcal{F},P,(\mathcal{F}_{t})_{t \in [0,T]})$を
    フィルトレーション付き確率空間とし,$(B(t))_{t \in [0,T]}$を
    $m$次元$\mathcal{F}_t$-ブラウン運動とする.
    $1\leq i\leq n, 1\leq r \leq m$に対して,Borel可測な函数
    $a^{i},\sigma_{r}^{i} \colon [0,T] \times \R^{n} \to \R$
    が与えられているとする.このとき,確率過程$(X(t))_{t \in [0,T]}$が
    $x \in R^{n}$を出発点とする確率微分方程式
    \begin{align}
        dX(t) = a(t,X(t)) dt + \sum_{r=1}^{m} \sigma_{r}(t,X(t)) d B^{r}(t)
    \end{align}
    あるいは成分ごとに書いた
    \begin{align}
        dX^{i} (t) = a^{i}(t,X(t)) dt + \sum_{r=1}^{m} \sigma_{r}^{i}(t,X(t)) dB^{r}(t)
    \end{align}
    の強解であるとは,以下の条件を満たすことをいう.
    \begin{itemize}
        \item $X(t)$は可測かつ$\mathcal{F}_t$-適合な連続確率過程である.
        \item 任意の$1\leq i\leq n$と$1\leq r\leq m$に対して,
        $\sigma_{r}^{i}(t,X(t)) \in \mathcal{L}^2(\mathcal{F}_t)$
        かつ$a(t,X(t)) \in L^1[0,T]$が満たされる.
        \item $X(t)$は確率積分方程式
        \begin{align}
            X(t) = x + \int_{0}^{t} a(s,X(s))ds + \int_{0}^{t} \sigma_{r}(s,X(s))dB^{r}(s)
        \end{align}
        あるいは成分ごとに書いた
        \begin{align}
            X^{i}(t) = x^{i} + \int_{0}^{t} a^{i}(s,X(s))ds + \int_{0}^{t} \sigma_{r}^{i}(s,X(s))dB^{r}(s)
        \end{align}
        を満たす.
    \end{itemize}
\end{definition}


強解の存在と一意性については次の定理がよく知られている.
証明は省略する.


\begin{theorem}\label{fundamental_existence}
    係数$a,\sigma_{r}$が以下を満たすと仮定する.
    \begin{itemize}
        \item Lipshitz連続,すなわち,
        \begin{align}
            \exists K>0,\quad \forall t \in [0,T],\quad \forall x,y \in \R^{n},\quad
            \abs{a(t,x)-a(t,y)} + \sum_{r=1}^{m} \abs{\sigma_{r}(t,x) - \sigma_{r}(t,y)} 
            \leq K\abs{x-y}
        \end{align}
        を満たす.
        \item 1次増大条件,すなわち
        \begin{align}
            \exists K>0,\quad \forall t \in [0,T],\quad \forall x\in \R^{n},\quad
            \abs{a(t,x)} + \sum_{r=1}^{m} \abs{\sigma_{r}(t,x)} \leq K(1+\abs{x})
        \end{align}
        を満たす.
    \end{itemize}
    このとき,確率微分方程式の強解$X(t)$で,各成分が$\mathcal{L}^2$に属するものが
    存在する.さらに
    $\tilde{X}(t)$も強解ならば,$P(\forall t \geq 0,X(t)=\tilde{X}(t))=1$
    が成り立つという意味で,解$X(t)$は一意である.
\end{theorem}


\subsection{数値スキームの導出}
以降,$\mathrm{Definition}\ref{strong_solution}$の確率微分方程式が
与えられ,その係数は$\mathrm{Theorem}\ref{fundamental_existence}$
の仮定を満たし,かつ十分になめらかであるとする.
初期値$x \in \R^n$に対して一意存在する強解を$X(t)$とする.
$0<h(<T)$に対して$t \in [0,T-h]$において
\begin{align}
    X(t+h) = X(t) + \int_{t}^{t+h} a(s,X(s))ds + 
    \sum_{r=1}^{m} \int_{t}^{t+h} \sigma_{r}(s,X(s))dB_{r}(s)
\end{align}
が成立するから,適当に積分を近似して数値スキームを導く.
確率積分の項をどう近似するかで大きく2種類に分かれる.


\subsubsection{Euler-Maruyama Scheme}
Drift項の近似は例えば陽的に
\begin{align}
    \int_{t}^{t+h} a(s,X(s))ds \approx a(t,X(t))h
\end{align}
とする.確率積分の項を
\begin{align}
    \sum_{r=1}^{m} \int_{t}^{t+h} \sigma_{r}(s,X(s))dB_{r}(s)
    \approx \sum_{r=1}^{m} \sigma_{r}(t,X(t)) (B_{r}(t+h) - B_{r}(t))
\end{align}
として近似する.これをもとにして,$Nh = T$となるような
$h>0$と$N \in \N$に対して,$t_{k}\coloneqq  kh$における数値解$X_{k}$
についての漸化式
\begin{align}
    \begin{cases}
        X_{k+1} = X_{k} + a(t_{k},X_{k})h 
        + \sum_{r=1}^{m} \sigma_{r}(t_{k},X_{k}) (B_{r}(t_{k+1}) - B_{r}(t_{k}))\\
        X_{0} = x
    \end{cases}
\end{align}
を得る.これを陽的Euler-Maruyamaスキームという.
$B_{r}(t_{k+1}) - B_{r}(t_{k})$は平均$0$で分散$h$の正規分布だから,
$\xi_{r}$を標準正規分布に従う確率変数として$B_{r}(t_{k+1}) - B_{r}(t_{k}) = \sqrt{h} \xi_{r}$
が成り立つ.これを使って書き直すと
\begin{align}
    X_{k+1} = X_{k} + a(t_{k},X_{k})h 
    + \sum_{r=1}^{m} \sigma_{r}(t_{k},X_{k}) \sqrt{h} \xi_{r} \label{Euler_Maruyama}
\end{align}
となる.以上と同様にしてdrift陰的Euler-Maruyamaが
\begin{align}
    X_{k+1} = X_{k} + a(t_{k+1},X_{k+1})h 
    + \sum_{r=1}^{m} \sigma_{r}(t_{k},X_{k}) \sqrt{h} \xi_{r}
\end{align}
のように与えられる.また,$\lambda \in (0,1)$に対し混合Euler-Maruyama
\begin{align}
    X_{k+1} = X_{k} + \lambda a(t_{k},X_{k})h + (1-\lambda) a(t_{k+1},X_{k+1})h 
    + \sum_{r=1}^{m} \sigma_{r}(t_{k},X_{k}) \sqrt{h} \xi_{r}
\end{align}
ないし
\begin{align}
    X_{k+1} = X_{k} + a(\lambda t_k + (1-\lambda)t_{k+1}, \lambda X_{k} + (1-\lambda)X_{k+1})h
    + \sum_{r=1}^{m} \sigma_{r}(t_{k},X_{k}) \sqrt{h} \xi_{r}
\end{align}
も与えられる.これらのスキームを比較するには安定性を調べなければならないが,それは別の機会にする.
\subsubsection{Milstein Scheme}
drift項の近似はEuler-Maruyamaと同様である.確率積分の近似が異なる.
\begin{align}
    \sum_{r=1}^{m} \int_{t}^{t+h} \sigma_{r}(s,X(s))dB_{r}(s)
\end{align}
において,被積分函数に伊藤の公式を適用して,
\begin{align}
    \sigma_{r}(s,X(s)) = \sigma_{r}(t,X(t)) 
    + \int_{t}^{s} \left( \frac{\partial \sigma_{r}}{\partial t}(u,X(u)) 
        + \frac{1}{2} \triangle \sigma_{r} (u,X(u)) \right)du \\
    + \sum_{l=1}^{m} \sum_{j=1}^{n} \int_{t}^{s} 
        \frac{\partial \sigma_{r}}{\partial x_{j}}(u,X(u)) 
        \sigma_{l}^{j}(u,X(u)) dB_{l}(u)
\end{align}
を得る.通常の積分の項は近似計算するときに$h^2$の項が出てくるので,$0$とみなす.すなわち,
\begin{align}
    \sigma_{r}(s,X(s)) \approx \sigma_{r}(t,X(t)) + 
    \sum_{l=1}^{m} \sum_{j=1}^{n} \int_{t}^{s} 
        \frac{\partial \sigma_{r}}{\partial x_{j}}(u,X(u)) 
        \sigma_{l}^{j}(u,X(u)) dB_{l}(u)
\end{align}
とする.これを代入して,
\begin{align}
    &\sum_{r=1}^{m} \int_{t}^{t+h} \sigma_{r}(s,X(s))dB_{r}(s) \\
    &= \sum_{r=1}^{m} \int_{t}^{t+h} \left( \sigma_{r}(t,X(t)) + 
    \sum_{l=1}^{m} \sum_{j=1}^{n} \int_{t}^{s} 
        \frac{\partial \sigma_{r}}{\partial x_{j}}(u,X(u)) 
        \sigma_{l}^{j}(u,X(u)) dB_{l}(u)  \right) dB_{r}(s)\\
    &\approx \sum_{r=1}^{m} \sigma_{r}(t,X(t)) (B(t+h)-B(t))
    + \sum_{r,l=1}^{m} \sum_{j=1}^{n} 
    \frac{\partial \sigma_{r}}{\partial x_{j}}(t,X(t)) \sigma_{l}^{j}(t,X(t)) 
    \int_{t}^{t+h} \int_{t}^{s} dB_{l}(u) dB_{r}(s)
\end{align}
を得る.これから,一般の陽的Milstein Schemeを
\begin{align}
    X_{k+1} &= X_{k} + a(t_{k},X_{k})h 
        + \sum_{r=1}^{m} \sigma_{r}(t_{k},X_{k}) (B(t+h)-B(t)) \\
        &\quad + \sum_{r,l=1}^{m} \sum_{j=1}^{n} 
        \frac{\partial \sigma_{r}}{\partial x_{j}}(t_{k},X_{k}) \sigma_{l}^{j}(t_{k},X_{k}) 
        \int_{t_{k}}^{t_{k+1}} \int_{t_{k}}^{s} dB_{l}(u) dB_{r}(s) \label{general_Milstein}
\end{align}
として導くことができる.$\int_{t}^{t+h} \int_{t}^{s} dB_{l}(u) dB_{r}(s)$は
一般に解析的に計算する方法が知られていない.さらにブラウン運動(正確には像測度を考えたWiener過程)
の汎函数として通常の広義一様収束位相では連続ではないため,近似計算も難しい.
ここでは解析的に計算ができるような場合を二つ紹介する.

$m=1$の場合.$\sigma_{1}(t,x)$を単に$\sigma(t,x)$とし,
$B_{1}(t)$を単に$B(t)$と書くことにする.
問題の項$\int_{t}^{t+h} \int_{t}^{s} dB_{l}(u) dB_{r}(s)$は
\begin{align}
    \int_{t}^{t+h} \int_{t}^{s} dB(u) dB(s) 
    &= \int_{t}^{t+h} B(s)-B(t) dB(s) \\
    &= \int_{t}^{t+h} B(s) dB(s) - B(t)(B(t+h)-B(t)) \\
    &= \frac{1}{2}\left(B(t+h)^{2} - B(t)^{2} - h\right) - B(t)(B(t+h)-B(t)) \\
    &= \frac{1}{2}\left((B(t+h)-B(t))^{2} -h\right)
\end{align}
として計算ができるので,Milstein Scheme$\eqref{general_Milstein}$は
\begin{align}
    X_{k+1} &= X_{k} + a(t_{k},X_{k})h 
     + \sigma(t_{k},X_{k})\sqrt{h}\xi + \frac{1}{2}
     \sum_{j=1}^{n} \frac{\partial \sigma}{\partial x_{j}}
     (t_{k},X_{k})\sigma^{j} (t_{k},X_{k}) (\xi^2 - 1)h
\end{align}
となる.ここで,$\xi$は標準正規分布に従う確率変数とした.
この表式のほうがどちらかというと有名だと思われる.

係数が対称な場合.任意の$1\leq l,r\leq n$に対して
\begin{align}
    \sum_{j=1}^{n} \frac{\partial \sigma_{r}}{\partial x_{j}} \sigma_{l}^{j}
    = \sum_{j=1}^{n} \frac{\partial \sigma_{l}}{\partial x_{j}} \sigma_{r}^{j}
\end{align}
という対称性があると仮定する.見やすさのために
$\Lambda_{r,l}\sigma (t,x) \coloneqq \sum_{j=1}^{n} 
\frac{\partial \sigma_{r}}{\partial x_{j}} (t,x) \sigma_{l}^{j}(t,x)$とおけば,
$\Lambda_{r,l}\sigma = \Lambda_{l,r}\sigma$となる.
このとき,
\begin{align}
    & \sum_{r,l=1}^{m} \sum_{j=1}^{n} 
        \frac{\partial \sigma_{r}}{\partial x_{j}}(t,X(t)) \sigma_{l}^{j}(t,X(t)) 
        \int_{t}^{t+h} \int_{t}^{s} dB_{l}(u) dB_{r}(s)\\
    &= \frac{1}{2} \sum_{r,l=1}^{m} \Lambda_{r,l}\sigma(t,X(t))
    \left(\int_{t}^{t+h} \int_{t}^{s} dB_{l}(u) dB_{r}(s) + 
    \int_{t}^{t+h} \int_{t}^{s} dB_{r}(u) dB_{l}(s)\right)
\end{align}
となる.この確率積分の項は,
\begin{align}
    &\int_{t}^{t+h} \int_{t}^{s} dB_{l}(u) dB_{r}(s) + 
    \int_{t}^{t+h} \int_{t}^{s} dB_{r}(u) dB_{l}(s) \\
    &= \int_{t}^{t+h} B_{r}(s) dB_{l}(s) + \int_{t}^{t+h} B_{l}(s) dB_{r}(s)
    - B_{r}(t)\left(B_{l}(t+h)-B_{l}(t)\right) - B_{l}(t)\left(B_{r}(t+h)-B_{r}(t)\right)\\
    &= B_{r}(t+h)B_{l}(t+h) - B_{r}(t)B_{l}(t)
    - B_{r}(t)(B_{l}(t+h)-B_{l}(t)) - B_{l}(t)(B_{r}(t+h)-B_{r}(t))\\
    &= (B_{r}(t+h)-B_{r}(t)) (B_{l}(t+h)-B_{l}(t))
\end{align}
として計算できる.以上により,
\begin{align}
    &\sum_{r,l=1}^{m} \sum_{j=1}^{n} 
        \frac{\partial \sigma_{r}}{\partial x_{j}}(t,X(t)) \sigma_{l}^{j}(t,X(t)) 
        \int_{t}^{t+h} \int_{t}^{s} dB_{l}(u) dB_{r}(s) \\
    &= \frac{1}{2} \sum_{r,l=1}^{m} \Lambda_{r,l}\sigma(t,X(t))
    (B_{r}(t+h)-B_{r}(t)) (B_{l}(t+h)-B_{l}(t))
\end{align}
を得る.よって,Milstein Scheme$\eqref{general_Milstein}$は
\begin{align}
    X_{k+1} = X_{k} + a(t_{k},X_{k})h 
    + \sum_{r=1}^{m} \sigma_{r}(t_{k},X_{k})\sqrt{h}\xi_{r}
     + \frac{1}{2} \sum_{r,l=1}^{m} \Lambda_{r,l}\sigma(t_{k},X_{k})
    \xi_{r} \xi_{l} h
\end{align}
となる.ただし,$1\leq r\leq m$に対して$\xi_{r}$は標準正規分布に従う確率変数とした.
drift陰的なスキームや混合スキームも同様にして得られるが,省略する.

\subsection{強収束}
定理の証明のため,One-Step Approximationを導入する.
\end{document}