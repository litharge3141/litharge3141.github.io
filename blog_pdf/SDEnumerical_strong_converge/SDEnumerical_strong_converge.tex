\documentclass[dvipdfmx,autodetect-engine]{jsarticle}


\usepackage[utf8]{inputenc}
\usepackage{amsmath}
\usepackage{amsthm}
\usepackage{amssymb}
\usepackage{mathtools}

\usepackage[dvipdfmx]{graphicx}

\newtheorem{theorem}{Theorem}[section]
\newtheorem{corollary}{Corollary}[theorem]
\newtheorem{lemma}[theorem]{Lemma}
\newtheorem{example}{Example}[section]

\theoremstyle{remark}
\newtheorem*{remark}{Remark}

\theoremstyle{definition}
\newtheorem{definition}{Definition}[section]
\mathtoolsset{showonlyrefs}

\renewcommand{\labelenumi}{(\arabic{enumi})}
\renewcommand{\labelenumii}{(\alph{enumii}}
\newcommand{\R}{\mathbb{R}}
\newcommand{\N}{\mathbb{N}}
\newcommand{\C}{\mathbb{C}}
\newcommand{\Z}{\mathbb{Z}}
\newcommand{\diver}{\mathrm{div} \,}
\newcommand{\rot}{\mathrm{rot} \,}
\newcommand{\abs}[1]{\left\lvert#1\right\rvert}
\newcommand{\norm}[1]{\left\lVert#1\right\rVert}
\newcommand{\setmid}{\mathrel{} \middle| \mathrel{}}


\begin{document}

\title{SDEの数値計算 強収束}
\author{@litharge3141}
\date{\today}
\maketitle

\abstract{}
SDEの数値計算の中で最も基本的なEuler-MaruyamaおよびMilsteinのスキームについて,
その厳密解への強収束と呼ばれる収束について述べる.
これは数値計算の各時間ステップについて,その厳密解からの分散が
刻み幅$h$を用いて上から評価できるというものである.

\section{数値スキームの強収束}

\subsection{準備}
主定理を述べる前に必要な用語などについて述べる.
約束事として,この文書を通して,フィルトレーション付き確率空間のフィルトレーションは右連続かつ
零集合をすべて含むものとする.


\begin{definition}
    $(\Omega,\mathcal{F},P,(\mathcal{F}_{t})_{t \in [0,T]})$を
    フィルトレーション付き確率空間とする.この確率空間上の$m$次元
    ブラウン運動$(B(t))_{t \in [0,T]}$が
    $\mathcal{F}_t$-ブラウン運動であるとは,以下の条件を満たすことをいう.
    \begin{itemize}
        \item $B(t)$は$\mathcal{F}_{t}$適合である.
        \item 任意の$0\leq s < t\leq T$に対して$B(t)-B(s)$は$\mathcal{F}_{s}$と独立.
    \end{itemize}
\end{definition}


\begin{definition}
    $(\Omega,\mathcal{F},P,(\mathcal{F}_{t})_{t \in [0,T]})$を
    フィルトレーション付き確率空間とする.
    $f\colon [0,T]\times \Omega \to \R$が可測であるとは,$f$が
    $\mathcal{B}([0,T])\otimes \mathcal{F} / \mathcal{B(\R)}$-可測であることをいう.
    \begin{align}
        \mathcal{L}^{2}(\mathcal{F}_{t}) \coloneqq
        \left\{f \setmid f \in L^{2}([0,T]\times \Omega),\,
        f \text{は}\mathcal{F}_{t}\text{-適合} \right\}
    \end{align}
    と定める.
\end{definition}

数値計算においては確率微分方程式の強解を近似して計算する.
\begin{definition}\label{strong_solution}
    $(\Omega,\mathcal{F},P,(\mathcal{F}_{t})_{t \in [0,T]})$を
    フィルトレーション付き確率空間とし,$(B(t))_{t \in [0,T]}$を
    $m$次元$\mathcal{F}_t$-ブラウン運動とする.
    $1\leq i\leq n, 1\leq r \leq m$に対して,Borel可測な函数
    $a^{i},\sigma_{r}^{i} \colon [0,T] \times \R^{n} \to \R$
    が与えられているとする.このとき,確率過程$(X(t))_{t \in [0,T]}$が
    $x \in \R^{n}$を出発点とする確率微分方程式
    \begin{align}
        dX(t) = a(t,X(t)) dt + \sum_{r=1}^{m} \sigma_{r}(t,X(t)) d B^{r}(t) \label{sde}
    \end{align}
    あるいは成分ごとに書いた
    \begin{align}
        dX^{i} (t) = a^{i}(t,X(t)) dt + \sum_{r=1}^{m} \sigma_{r}^{i}(t,X(t)) dB^{r}(t)
    \end{align}
    の強解であるとは,以下の条件を満たすことをいう.
    \begin{itemize}
        \item $X(t)$は可測かつ$\mathcal{F}_t$-適合な連続確率過程である.
        \item 任意の$1\leq i\leq n$と$1\leq r\leq m$に対して,
        $\sigma_{r}^{i}(t,X(t)) \in \mathcal{L}^2(\mathcal{F}_t)$
        かつ$a(t,X(t)) \in L^1[0,T]$が満たされる.
        \item $X(t)$は確率積分方程式
        \begin{align}
            X(t) = x + \int_{0}^{t} a(s,X(s))ds + \int_{0}^{t} \sigma_{r}(s,X(s))dB^{r}(s)
        \end{align}
        あるいは成分ごとに書いた
        \begin{align}
            X^{i}(t) = x^{i} + \int_{0}^{t} a^{i}(s,X(s))ds + \int_{0}^{t} \sigma_{r}^{i}(s,X(s))dB^{r}(s)
        \end{align}
        を満たす.
    \end{itemize}
\end{definition}


強解の存在と一意性については次の定理がよく知られている.
証明は省略する.


\begin{theorem}\label{fundamental_existence}
    係数$a,\sigma_{r}$が以下を満たすと仮定する.
    \begin{itemize}
        \item Lipshitz連続,すなわち,
        \begin{align}
            \exists K>0,\quad \forall t \in [0,T],\quad \forall x,y \in \R^{n},\quad
            \abs{a(t,x)-a(t,y)} + \sum_{r=1}^{m} \abs{\sigma_{r}(t,x) - \sigma_{r}(t,y)} 
            \leq K\abs{x-y}
        \end{align}
        を満たす.
        \item 1次増大条件,すなわち
        \begin{align}
            \exists K>0,\quad \forall t \in [0,T],\quad \forall x\in \R^{n},\quad
            \abs{a(t,x)} + \sum_{r=1}^{m} \abs{\sigma_{r}(t,x)} \leq K(1+\abs{x})
        \end{align}
        を満たす.
    \end{itemize}
    このとき,確率微分方程式の強解$X(t)$で,各成分が$\mathcal{L}^2 (\mathcal{F}_{t})$に属するものが
    存在する.さらに
    $\tilde{X}(t)$も強解ならば,$P(\forall t \geq 0,X(t)=\tilde{X}(t))=1$
    が成り立つという意味で,解$X(t)$は一意である.
\end{theorem}


特に初期値が確率変数の場合,次が知られている.
数値計算のように確率微分方程式を時間の区間ごとに区切って考えた場合,
各区間ごとの確率微分方程式の(数値)解を初期値として次の時間ステップの解を
評価することになるから,この形の定理が必要になる.


\begin{theorem}\label{initial_random_existence}
    $\mathcal{F}_{0}$可測な$\R^{n}$に値を取る確率変数$X_{0}$が与えられ,
    $E[\abs{X_{0}}^{2}] < \infty$を満たすとする.
    係数$a,\sigma_{r}$が以下を満たすと仮定する.
    \begin{itemize}
        \item Lipshitz連続,すなわち,
        \begin{align}
            \exists K>0,\quad \forall t \in [0,T],\quad \forall x,y \in \R^{n},\quad
            \abs{a(t,x)-a(t,y)} + \sum_{r=1}^{m} \abs{\sigma_{r}(t,x) - \sigma_{r}(t,y)} 
            \leq K\abs{x-y}
        \end{align}
        を満たす.
        \item 1次増大条件,すなわち
        \begin{align}
            \exists K>0,\quad \forall t \in [0,T],\quad \forall x\in \R^{n},\quad
            \abs{a(t,x)} + \sum_{r=1}^{m} \abs{\sigma_{r}(t,x)} \leq K(1+\abs{x})
        \end{align}
        を満たす.
    \end{itemize}
    このとき,確率微分方程式の強解,すなわち各成分が確率積分方程式
    \begin{align}
        X^{i}(t) = X_{0}^{i} + \int_{0}^{t} a^{i}(s,X(s))ds +
         \int_{0}^{t} \sigma_{r}^{i}(s,X(s))dB^{r}(s)
    \end{align}
    を満たす確率過程$X(t)$で,$\mathrm{Definition}\ref{strong_solution}$
    の仮定を満たすものが存在する.
    さらに,$X(t)$は$\mathcal{L}^2 (\mathcal{F}_{t})$に属していて,
    $\tilde{X}(t)$も強解ならば,$P(\forall t \geq 0,X(t)=\tilde{X}(t))=1$
    が成り立つという意味で,一意である.
\end{theorem}


\subsection{数値スキームの導出}
この節では,$\mathrm{Definition}\ref{strong_solution}$の確率微分方程式が
与えられ,その係数は$\mathrm{Theorem}\ref{fundamental_existence}$
の仮定を満たし,かつ十分になめらかであるとする.
初期値$x \in \R^n$に対して一意存在する強解を$X(t)$とする.
$0<h(<T)$に対して$t \in [0,T-h]$において
\begin{align}
    X(t+h) = X(t) + \int_{t}^{t+h} a(s,X(s))ds + 
    \sum_{r=1}^{m} \int_{t}^{t+h} \sigma_{r}(s,X(s))dB^{r}(s)
\end{align}
が成立するから,適当に積分を近似して数値スキームを導く.
確率積分の項をどう近似するかで大きく2種類に分かれる.


\subsubsection{Euler-Maruyama Scheme}
Drift項の近似は例えば陽的に
\begin{align}
    \int_{t}^{t+h} a(s,X(s))ds \approx a(t,X(t))h
\end{align}
とする.確率積分の項を
\begin{align}
    \sum_{r=1}^{m} \int_{t}^{t+h} \sigma_{r}(s,X(s))dB^{r}(s)
    \approx \sum_{r=1}^{m} \sigma_{r}(t,X(t)) (B^{r}(t+h) - B^{r}(t))
\end{align}
として近似する.これをもとにして,$Nh = T$となるような
$h>0$と$N \in \N$に対して,$t_{k}\coloneqq  kh$における数値解$X_{k}$
についての漸化式
\begin{align}
    \begin{cases}
        X_{k+1} = X_{k} + a(t_{k},X_{k})h 
        + \sum_{r=1}^{m} \sigma_{r}(t_{k},X_{k}) (B^{r}(t_{k+1}) - B^{r}(t_{k}))\\
        X_{0} = x
    \end{cases}
\end{align}
を得る.これを陽的Euler-Maruyamaスキームという.
$B^{r}(t_{k+1}) - B^{r}(t_{k})$は平均$0$で分散$h$の正規分布だから,
$\xi^{r}$を標準正規分布に従う確率変数として$B^{r}(t_{k+1}) - B^{r}(t_{k}) = \sqrt{h} \xi^{r}$
が成り立つ.これを使って書き直すと
\begin{align}
    X_{k+1} = X_{k} + a(t_{k},X_{k})h 
    + \sum_{r=1}^{m} \sigma_{r}(t_{k},X_{k}) \sqrt{h} \xi^{r} \label{Euler_Maruyama}
\end{align}
となる.以上と同様にしてdrift陰的Euler-Maruyamaが
\begin{align}
    X_{k+1} = X_{k} + a(t_{k+1},X_{k+1})h 
    + \sum_{r=1}^{m} \sigma_{r}(t_{k},X_{k}) \sqrt{h} \xi^{r}
\end{align}
のように与えられる.また,$\lambda \in (0,1)$に対し混合Euler-Maruyama
\begin{align}
    X_{k+1} = X_{k} + \lambda a(t_{k},X_{k})h + (1-\lambda) a(t_{k+1},X_{k+1})h 
    + \sum_{r=1}^{m} \sigma_{r}(t_{k},X_{k}) \sqrt{h} \xi^{r}
\end{align}
ないし
\begin{align}
    X_{k+1} = X_{k} + a(\lambda t_k + (1-\lambda)t_{k+1}, \lambda X_{k} + (1-\lambda)X_{k+1})h
    + \sum_{r=1}^{m} \sigma_{r}(t_{k},X_{k}) \sqrt{h} \xi^{r}
\end{align}
も与えられる.これらのスキームを比較するには安定性を調べなければならないが,それは別の機会にする.
\subsubsection{Milstein Scheme}
drift項の近似はEuler-Maruyamaと同様である.確率積分の近似が異なる.
\begin{align}
    \sum_{r=1}^{m} \int_{t}^{t+h} \sigma_{r}(s,X(s))dB^{r}(s)
\end{align}
において,被積分函数に伊藤の公式を適用して,
\begin{align}
    \sigma_{r}(s,X(s)) = \sigma_{r}(t,X(t)) 
    + \int_{t}^{s} \left( \frac{\partial \sigma_{r}}{\partial t}(u,X(u)) 
        + \frac{1}{2} \triangle \sigma_{r} (u,X(u)) \right)du \\
    + \sum_{l=1}^{m} \sum_{j=1}^{n} \int_{t}^{s} 
        \frac{\partial \sigma_{r}}{\partial x_{j}}(u,X(u)) 
        \sigma_{l}^{j}(u,X(u)) dB^{l}(u)
\end{align}
を得る.通常の積分の項は近似計算するときに$h^2$の項が出てくるので,$0$とみなす.すなわち,
\begin{align}
    \sigma_{r}(s,X(s)) \approx \sigma_{r}(t,X(t)) + 
    \sum_{l=1}^{m} \sum_{j=1}^{n} \int_{t}^{s} 
        \frac{\partial \sigma_{r}}{\partial x_{j}}(u,X(u)) 
        \sigma_{l}^{j}(u,X(u)) dB^{l}(u)
\end{align}
とする.これを代入して,
\begin{align}
    &\sum_{r=1}^{m} \int_{t}^{t+h} \sigma_{r}(s,X(s))dB^{r}(s) \\
    &= \sum_{r=1}^{m} \int_{t}^{t+h} \left( \sigma_{r}(t,X(t)) + 
    \sum_{l=1}^{m} \sum_{j=1}^{n} \int_{t}^{s} 
        \frac{\partial \sigma_{r}}{\partial x_{j}}(u,X(u)) 
        \sigma_{l}^{j}(u,X(u)) dB^{l}(u)  \right) dB^{r}(s)\\
    &\approx \sum_{r=1}^{m} \sigma_{r}(t,X(t)) (B(t+h)-B(t))
    + \sum_{r,l=1}^{m} \sum_{j=1}^{n} 
    \frac{\partial \sigma_{r}}{\partial x_{j}}(t,X(t)) \sigma_{l}^{j}(t,X(t)) 
    \int_{t}^{t+h} \int_{t}^{s} dB^{l}(u) dB^{r}(s)
\end{align}
を得る.これから,一般の陽的Milstein Schemeを
\begin{align}
    X_{k+1} &= X_{k} + a(t_{k},X_{k})h 
        + \sum_{r=1}^{m} \sigma_{r}(t_{k},X_{k}) (B(t+h)-B(t)) \\
        &\quad + \sum_{r,l=1}^{m} \sum_{j=1}^{n} 
        \frac{\partial \sigma_{r}}{\partial x_{j}}(t_{k},X_{k}) \sigma_{l}^{j}(t_{k},X_{k}) 
        \int_{t_{k}}^{t_{k+1}} \int_{t_{k}}^{s} dB^{l}(u) dB^{r}(s) \label{general_Milstein}
\end{align}
として導くことができる.$\int_{t}^{t+h} \int_{t}^{s} dB^{l}(u) dB^{r}(s)$は
一般に解析的に計算する方法が知られていない.さらにブラウン運動(正確には像測度を考えたWiener過程)
の汎函数として通常の広義一様収束位相では連続ではないため,近似計算も難しい.
ここでは解析的に計算ができるような場合を二つ紹介する.

$m=1$の場合.$\sigma_{1}(t,x)$を単に$\sigma(t,x)$とし,
$B_{1}(t)$を単に$B(t)$と書くことにする.
問題の項$\int_{t}^{t+h} \int_{t}^{s} dB^{l}(u) dB^{r}(s)$は
\begin{align}
    \int_{t}^{t+h} \int_{t}^{s} dB(u) dB(s) 
    &= \int_{t}^{t+h} B(s)-B(t) dB(s) \\
    &= \int_{t}^{t+h} B(s) dB(s) - B(t)(B(t+h)-B(t)) \\
    &= \frac{1}{2}\left(B(t+h)^{2} - B(t)^{2} - h\right) - B(t)(B(t+h)-B(t)) \\
    &= \frac{1}{2}\left((B(t+h)-B(t))^{2} -h\right)
\end{align}
として計算ができるので,Milstein Scheme$\eqref{general_Milstein}$は
\begin{align}
    X_{k+1} &= X_{k} + a(t_{k},X_{k})h 
     + \sigma(t_{k},X_{k})\sqrt{h}\xi + \frac{1}{2}
     \sum_{j=1}^{n} \frac{\partial \sigma}{\partial x_{j}}
     (t_{k},X_{k})\sigma^{j} (t_{k},X_{k}) (\xi^2 - 1)h
\end{align}
となる.ここで,$\xi$は標準正規分布に従う確率変数とした.
この表式のほうがどちらかというと有名だと思われる.

係数が対称な場合.任意の$1\leq l,r\leq n$に対して
\begin{align}
    \sum_{j=1}^{n} \frac{\partial \sigma_{r}}{\partial x_{j}} \sigma_{l}^{j}
    = \sum_{j=1}^{n} \frac{\partial \sigma_{l}}{\partial x_{j}} \sigma_{r}^{j}
\end{align}
という対称性があると仮定する.見やすさのために
$\Lambda_{r,l}\sigma (t,x) \coloneqq \sum_{j=1}^{n} 
\frac{\partial \sigma_{r}}{\partial x_{j}} (t,x) \sigma_{l}^{j}(t,x)$とおけば,
$\Lambda_{r,l}\sigma = \Lambda_{l,r}\sigma$となる.
このとき,
\begin{align}
    & \sum_{r,l=1}^{m} \sum_{j=1}^{n} 
        \frac{\partial \sigma_{r}}{\partial x_{j}}(t,X(t)) \sigma_{l}^{j}(t,X(t)) 
        \int_{t}^{t+h} \int_{t}^{s} dB^{l}(u) dB^{r}(s)\\
    &= \frac{1}{2} \sum_{r,l=1}^{m} \Lambda_{r,l}\sigma(t,X(t))
    \left(\int_{t}^{t+h} \int_{t}^{s} dB^{l}(u) dB^{r}(s) + 
    \int_{t}^{t+h} \int_{t}^{s} dB^{r}(u) dB^{l}(s)\right)
\end{align}
となる.この確率積分の項は,
\begin{align}
    &\int_{t}^{t+h} \int_{t}^{s} dB^{l}(u) dB^{r}(s) + 
    \int_{t}^{t+h} \int_{t}^{s} dB^{r}(u) dB^{l}(s) \\
    &= \int_{t}^{t+h} B^{r}(s) dB^{l}(s) + \int_{t}^{t+h} B^{l}(s) dB^{r}(s)\\
    &\quad - B^{r}(t)(B^{l}(t+h)-B^{l}(t)) - B^{l}(t)(B^{r}(t+h)-B^{r}(t))\\
    &= B^{r}(t+h)B^{l}(t+h) - B^{r}(t)B^{l}(t) \\
    &\quad - B^{r}(t)(B^{l}(t+h)-B^{l}(t)) - B^{l}(t)(B^{r}(t+h)-B^{r}(t))\\
    &= (B^{r}(t+h)-B^{r}(t)) (B^{l}(t+h)-B^{l}(t))
\end{align}
として計算できる.以上により,
\begin{align}
    &\sum_{r,l=1}^{m} \sum_{j=1}^{n} 
        \frac{\partial \sigma_{r}}{\partial x_{j}}(t,X(t)) \sigma_{l}^{j}(t,X(t)) 
        \int_{t}^{t+h} \int_{t}^{s} dB^{l}(u) dB^{r}(s) \\
    &= \frac{1}{2} \sum_{r,l=1}^{m} \Lambda_{r,l}\sigma(t,X(t))
    (B^{r}(t+h)-B^{r}(t)) (B^{l}(t+h)-B^{l}(t))
\end{align}
を得る.よって,Milstein Scheme$\eqref{general_Milstein}$は
\begin{align}
    X_{k+1} = X_{k} + a(t_{k},X_{k})h 
    + \sum_{r=1}^{m} \sigma_{r}(t_{k},X_{k})\sqrt{h}\xi^{r}
     + \frac{1}{2} \sum_{r,l=1}^{m} \Lambda_{r,l}\sigma(t_{k},X_{k})
    \xi^{r} \xi^{l} h
\end{align}
となる.ただし,$1\leq r\leq m$に対して$\xi^{r}$は標準正規分布に従う確率変数とした.
drift陰的なスキームや混合スキームも同様にして得られるが,省略する.

\subsection{強収束}
この節では,$\mathrm{Definition}\ref{strong_solution}$の確率微分方程式が
与えられ,$\mathrm{Theorem}\ref{initial_random_existence}$
の仮定を満たす係数と初期値が与えられ,係数は十分になめらかであるとする.
初期値$X_{0}$に対して一意存在する強解を$X(t)$とする.
数値解の収束の概念を述べる.
通常の微分方程式とは異なり,モーメントの収束についての定義になる.


\begin{definition}
    数値スキームが$L^p$において$\gamma$次強収束するとは,
    ある$K>0$が存在して,十分小さい任意の$h>0$と任意の$Nh=T$となる$N \in \N$と
    任意の$0\leq k\leq N$に対して,確率微分方程式の解$X(t_{k})$と
    $t=t_{k}$における数値解$X_{k}$についての不等式
    \begin{align}
        E[\abs{X(t_{k}) - X_{k}}^{p}]^{\frac{1}{p}} \leq Kh^{\gamma}
    \end{align}
    が成り立つことをいう.
\end{definition}


収束定理の主張と証明のため,one-step approximationを導入する.
\begin{definition}[one-step approximation]
    $t\in [0,T]$と$x \in \R^{n}$および数値スキームが与えられているとする.
    このとき,初期条件を$X(t)=x$とする確率微分方程式$\eqref{sde}$の強解$X$の時刻$t+h$での値を
    $X_{t,x}(t+h)$と書くことにする.また,初期条件を$X(t)=x$とする確率微分方程式$\eqref{sde}$の
    時刻$t+h$での数値解を$\bar{X}_{t,x}(t+h)$と書くことにする.
    $\bar{X}_{t,x}(t+h)$を$X_{t,x}(t+h)$のone-step approximationという.
\end{definition}


この定義を用いると陽解法における数値解は次のように書き直せる.
\begin{align}
    X_{k+1} &= \bar{X}_{t_{k},X_{k}}(t_{k+1}) \\
            &= X_{k} + A\left(t_{k},X_{k},h,B^{r}(\theta)-B^{r}(t_{k})
            ; 1\leq r\leq m ,t_{k} \leq \theta \leq t_{k+1} \right)
\end{align}
ここで$A$は数値スキームによって異なる関数で,陽的Euler-Maruyamaであれば
\begin{align}
    A\left(t_{k},X_{k},h,B^{r}(\theta)-B^{r}(t_{k})
    ; 1\leq r\leq m ,t_{k} \leq \theta \leq t_{k+1} \right) \\
    = a(t_{k},X_{k})h + \sum_{r=1}^{m}\sigma_{r}(t_{k},X_{k})(B^{r}(t_{k+1})-B^{r}(t_{k}))
\end{align}
となる.強収束を主張する定理を述べる.


\begin{theorem}\label{fundamental_thm_for_osa}
    確率微分方程式$\eqref{sde}$を$\mathrm{Theorem}\ref{initial_random_existence}$
    の仮定を満たすような初期条件$X(0)=X_{0}$の下で考える.
    数値スキームが与えられているとする.
    ある$q_{2}\geq 1/2,\, q_{1} \geq q_{2} + 1/2 ,\, 
    p\geq 1,\, \alpha \geq 1,\, K>0$が存在して,
    十分小さい任意の$h>0$と任意の$0\leq t\leq T-h,\, x\in \R^{n}$
    に対してone-step approximationの誤差の評価
    \begin{align}
        \abs{E[X_{t,x}(t+h) - \bar{X}_{t,x}(t+h)]} 
        \leq K(1+\abs{x}^{2})^{\frac{1}{2}} h^{q_{1}} \\
        E[\abs{X_{t,x}(t+h)- \bar{X}_{t,x}(t+h)}^{2p}]^{\frac{1}{2p}}
        \leq K(1+\abs{x}^{2p})^{\frac{1}{2p}} h^{q_{2}}
    \end{align}
    が成り立つと仮定する.このとき,任意の$Nh=T$となる
    $N \in \N$と任意の$0\leq k\leq N$に
    対して,$k,h$によらないある定数$M>0$が存在して,
    \begin{align}
        E[\abs{X(t_{k}) - X_{k}}^{2p}]^{\frac{1}{2p}}
        \leq M(1+E[\abs{X_{0}}^{2p}])^{\frac{1}{2p}} h^{q_{2} - \frac{1}{2}}
    \end{align}
    が成り立つ.特に,この数値スキームは$q_{2}-1/2$次の強収束である.
\end{theorem}


Euler-MaruyamaやMilsteinスキームの$L^2$における強収束次数を求めるには$p=1$
とすれば十分なので,$p=1$の場合に証明する.補題を準備する.


\begin{lemma}[確率微分方程式の初期値に対する安定性]\label{sde_stability_initial}
    $x,y \in \R^{n}$とする.十分小さい任意の$h>0$と任意の$t \in [0,T-h]$に対する
    one-step approximationについて,
    ある確率過程$Z$が存在して,次のように表せる.
    \begin{align}
        X_{t,x}(t+h)-X_{t,y}(t+h) = x - y + Z
    \end{align}
    さらに,$h$に依存しない定数$K_{1},K_{2}>0$が存在して,次の評価が成り立つ.
    \begin{align}
        &E[\abs{X_{t,x}(t+h)-X_{t,y}(t+h)}^{2}] \leq
        \abs{x-y}^{2} (1+K_{1}h) \\
        &E[Z^{2}] \leq \abs{x-y}^{2} K_{2}h
    \end{align}
\end{lemma}


\begin{proof}
    伊藤の公式から$0\leq \theta \leq h$に対して
    \begin{align}
        &\abs{X_{t,x}(t+\theta)-X_{t,y}(t+\theta)}^{2} \\
        &= \abs{x-y}^{2} + 
        2\int_{t}^{t+\theta} (X_{t,x}(s)-X_{t,y}(s))\cdot (a(s,X_{t,x}(s))-a(s,X_{t,y}(s)))ds \\
        &+ 2\int_{t}^{t+\theta} (X_{t,x}(s)-X_{t,y}(s)) 
        \cdot \sum_{r=1}^{m} (\sigma_{r}(s,X_{t,x}(s))-\sigma_{r}(s,X_{t,x}(s))) dB^{r}(s)\\
        &+ \int_{t}^{t+\theta} \sum_{r=1}^{m} 
        \abs{\sigma_{r}(s,X_{t,x}(s))-\sigma_{r}(s,X_{t,y}(s))}^{2} ds
    \end{align}
    を得る.両辺の期待値を取ると確率積分の項が消えて,
    \begin{align}
        &E[\abs{X_{t,x}(t+\theta)-X_{t,y}(t+\theta)}^{2}] \\
        &= \abs{x-y}^{2} + 
        2E\left[\int_{t}^{t+\theta} (X_{t,x}(s)-X_{t,y}(s))
        \cdot (a(s,X_{t,x}(s))-a(s,X_{t,y}(s)))ds \right]\\
        &+ E\left[
            \int_{t}^{t+\theta} \sum_{r=1}^{m} 
        \abs{\sigma_{r}(s,X_{t,x}(s))-\sigma_{r}(s,X_{t,y}(s))}^{2} ds
        \right]
    \end{align}
    となる.Schwarzの不等式と係数のLipshitz条件から,$K>0$が存在して
    \begin{align}
        &2E\left[\int_{t}^{t+\theta} (X_{t,x}(s)-X_{t,y}(s))
        \cdot (a(s,X_{t,x}(s))-a(X_{t,y}(s)))ds \right]\\
        &\leq 2K \int_{t}^{t+\theta} E[\abs{X_{t,x}(s)-X_{t,y}(s)}^{2}]
    \end{align}
    および
    \begin{align}
        &E\left[
            \int_{t}^{t+\theta} \sum_{r=1}^{m} 
        \abs{\sigma_{r}(s,X_{t,x}(s))-\sigma_{r}(s,X_{t,y}(s))}^{2} ds
        \right]\\
        &\leq K \int_{t}^{t+\theta} E[\abs{X_{t,x}(s)-X_{t,y}(s)}^{2}]
    \end{align}
    を得る.したがって不等式
    \begin{align}
        E[\abs{X_{t,x}(t+\theta)-X_{t,y}(t+\theta)}^{2}] \leq
        \abs{x-y}^{2} + 3K \int_{t}^{t+\theta} E[\abs{X_{t,x}(s)-X_{t,y}(s)}^{2}]
    \end{align}
    が得られた.Gronwallの不等式からある$K_{1}>0$に対して
    \begin{align}
        E[\abs{X_{t,x}(t+\theta)-X_{t,y}(t+\theta)}^{2}]
        &\leq \abs{x-y}^{2} e^{3K \theta }\\
        &\leq \abs{x-y}^{2} (1+K_{1}\theta)
    \end{align}
    となり,求める不等式が得られる.
    \begin{align}
        Z &= \int_{t}^{t+\theta} a(s,X_{t,x}(s))-a(X_{t,y}(s)) ds \\
        &+ \int_{t}^{t+\theta} \sum_{r=1}^{m} 
        \sigma_{r}(s,X_{t,x}(s))-\sigma_{r}(s,X_{t,y}(s)) dB^{r}(s)
    \end{align}
    として同様にすれば$Z$についての不等式も得られる.
\end{proof}


\begin{lemma}[conditional estimate]
    $X,Y$を$\mathcal{F}_{t_{k}}$-可測な確率変数で$E[\abs{X}^{2}]<\infty$
    および$E[\abs{Y}^{2}]<\infty$が成り立つものとする.
    $\mathrm{Theorem}\ref{fundamental_thm_for_osa}$の仮定が成り立つとき,
    \begin{align}
        E\left[ \abs{ 
            E [ 
                X_{t,X}(t+h) - \bar{X}_{t,X}(t+h) \setmid \mathcal{F}_{t_{k}} 
            ] 
            } 
        \right]
        \leq K(
            1+E[
            \abs{X}^{2}
            ]
            )^{\frac{1}{2}} h^{q_{1}} \\
        E\left[ 
            E[
                \abs{ X_{t,X}(t+h)- \bar{X}_{t,X}(t+h) }^{2p}
            \mid \mathcal{F}_{t_{k}}
        ]
        \right]^{\frac{1}{2p}}
        \leq K(1+E[\abs{X}^{2p}])^{\frac{1}{2p}} h^{q_{2}}
    \end{align}
    となる.また,$\mathrm{Lemma}\ref{sde_stability_initial}$について
    \begin{align}
        &X_{t,X}(t+h) - X_{t,Y}(t+h) = X-Y + Z \\
        &E\left[
            E[
                \abs{
                    X_{t,X}(t+h) - X_{t,Y}(t+h) 
                }^{2}
                \setmid \mathcal{F}_{t_{k}}
            ]
        \right]
        \leq
        E[
            \abs{X-Y}^{2}
        ]
        (1+K_{1}h) \\
        &E\left[
            E[
                Z^{2} \setmid \mathcal{F}_{t_{k}}
            ]
        \right]
        \leq
        E[
            \abs{X-Y}^{2}
        ] K_{2}h
    \end{align}
    が成り立つ.
\end{lemma}


\begin{lemma}[数値解の初期値に対する評価]
    ある$K>0$が存在して,任意の$N \in \N$と任意の$0\leq k\leq N$に対して,
    \begin{align}
        E[\abs{\bar{X}_{k}}^{2}] \leq K(1+E[\abs{X_{0}}^{2}])
    \end{align}
    が成立する.
\end{lemma}


\begin{lemma}
    非負実数列$u_{k}$がある定数$A,B\geq 0,\,p\geq 1$と
    任意の$N \in \N$と任意の$0\leq k\leq N$に対して,$h = T/N$として
    \begin{align}
        u_{k+1} \leq (1+Ah)u_{k} + Bh^{p}
    \end{align}
    を満たすと仮定する.このとき,
    任意の$0\leq k\leq N$に対して,
    \begin{align}
        u_{k} \leq e^{AT}u_{0} + \frac{B}{A}(e^{AT}-1)h^{p-1}
    \end{align}
    が成立する.ただし,$A=0$の時は
    \begin{align}
        u_{k} \leq u_{0} + \frac{B}Th^{p-1}
    \end{align}
    とする.
\end{lemma}


\begin{proof}
    $A\neq 0$のときに示す.$A=0$のときも同様である.
    漸化式を繰り返し用いると,$0\leq k\leq N$に対して
    \begin{align}
        u_{k} \leq (1+Ah)^{k}u_{0} + Bh^{p} \frac{(1+Ah)^{k}-1}{Ah}
    \end{align}
    を得る.$(1+Ah)^{k} \leq (1+Ah)^{N}$であり,$x\geq 0$において
    $(1+x)^{N} \leq e^{xN}$が成り立つことから$(1+Ah)^{N} \leq e^{AhN} = e^{AT}$
    が成立する.したがって
    \begin{align}
        u_{k} &\leq e^{AT} u_{0} + Bh^{p} \frac{e^{AT}-1}{Ah} \\
        &\leq e^{AT}u_{0} + \frac{B}{A}(e^{AT}-1)h^{p-1}
    \end{align}
    となり,示された.
\end{proof}
\end{document}