\documentclass[dvipdfmx,autodetect-engine]{jsarticle}


\usepackage[utf8]{inputenc}
\usepackage{amsmath}
\usepackage{amsthm}
\usepackage{amssymb}
\usepackage{mathtools}
\usepackage{stix}

\usepackage[dvipdfmx]{graphicx}

\newtheorem{theorem}{Theorem}[section]
\newtheorem{corollary}{Corollary}[theorem]
\newtheorem{lemma}[theorem]{Lemma}
\newtheorem{example}{Example}[section]
\newtheorem*{ans}{解答}

\theoremstyle{remark}
\newtheorem*{remark}{Remark}

\theoremstyle{definition}
\newtheorem{definition}{Definition}[section]
\mathtoolsset{showonlyrefs}
\allowdisplaybreaks

\renewcommand{\labelenumi}{(\arabic{enumi})}
\renewcommand{\labelenumii}{(\alph{enumii}}
\newcommand{\R}{\mathbb{R}}
\newcommand{\N}{\mathbb{N}}
\newcommand{\C}{\mathbb{C}}
\newcommand{\Z}{\mathbb{Z}}
\newcommand{\diver}{\mathrm{div} \,}
\newcommand{\rot}{\mathrm{rot} \,}
\newcommand{\abs}[1]{\left\lvert#1\right\rvert}
\newcommand{\norm}[1]{\left\lVert#1\right\rVert}
\newcommand{\setmid}{\mathrel{} \middle| \mathrel{}}
\newcommand{\paren}[1]{\left( #1 \right)}
\newcommand{\iprod}[1]{\left\langle #1 \right\rangle}


\begin{document}

\title{Evans 演習問題解答}
\author{@litharge3141}
\date{\today}
\maketitle

\abstract
Evans, Partial Differential Equationsの演習問題の解答.
問題は載せません.

\section{1章の問題}
略
\section{2章の問題}
\subsection{方針と解答}
$cu$がなければ解けるので$cu$を非斉次項だと思って定数変化法を用いる.
\begin{ans}
    $v(t,x) = g(x-bt)$とおく.
    $v_{t} = - b \cdot Dv$が満たされることに注意する.
    $u(t,x) = \varphi(t) v(t,x)$とおいて方程式に代入すると
    \begin{align}
        &\varphi'(t) v(t,x) + \varphi(t) \partial_{t} v(t,x) + b\cdot Du(t,x) + c
        \varphi(t)v(t,x)\\
        &=\varphi'(t) v(t,x)+ c\varphi(t)v(t,x) = 0
    \end{align}
    から$\varphi'(t) = -c \varphi(t)$を得る.
    よって$\varphi(t) = Ae^{-ct}$となり,初期条件と合わせて
    $u(t,x) = e^{-ct}g(x-bt)$を得る.
\end{ans}


\subsection{方針と解答}
公式を導くつもりで成分計算をする.$O$は大文字と小文字の見分けがつかないので別の記号にしてほしい……
\begin{ans}
    $O$が直交行列であることから,任意の$1\leq i,j \leq n$に対して
    $\sum_{k=1}^{n} O_{ik} O'_{kj} = \sum_{k=1}^{n} O_{ik} O_{jk} = \delta_{ij}$
    が成り立つことに注意する.ここで$\delta_{ij}$は$i=j$のとき$1$でそれ以外は$0$として
    定める(単位行列の$i,j$成分).
    $u$が調和関数であると仮定する.
    $1\leq s\leq n$に対して,
    \begin{equation}
        \frac{\partial}{\partial x_{s}} u(Ox)
        = \sum_{i=1}^{n} u_{x_{i}}(Ox) \frac{\partial (Ox)_{i}}{\partial x_{s}}
    \end{equation}
    となる.$(Ox)_{i} = \sum_{j=1}^{n} O_{ij}x_{j}$となるから
    $\frac{\partial (Ox)_{i}}{\partial x_{s}} = O_{is}$となる.
    したがって
    \begin{equation}
        \frac{\partial^{2}}{\partial x_{s}^{2}} u(Ox)
        = \frac{\partial}{\partial x_{s}}
        \sum_{i=1}^{n} u_{x_{i}}(Ox) O_{is} 
        = \sum_{j=1}^{n} \sum_{i=1}^{n} u_{x_{i}x_{j}}(Ox) O_{is} O_{js}
    \end{equation}
    となるから,
    \begin{align}
        &\triangle u(Ox) = \sum_{s=1}^{n}  \frac{\partial^{2}}{\partial x_{s}^{2}} u(Ox)
        = \sum_{j=1}^{n} \sum_{i=1}^{n} u_{x_{i}x_{j}}(Ox) \sum_{s=1}^{n}O_{is} O_{js}
        = \sum_{j=1}^{n} \sum_{i=1}^{n} u_{x_{i}x_{j}}(Ox) \delta_{ij} \\
        &= \sum_{i=1}^{n} u_{x_{i}x_{i}}(Ox) = 0\quad(\because u\text{は調和関数})
    \end{align}
    により結論を得た.
\end{ans}


\subsection{方針と解答}
示すべき式の右辺は$r$に依存するが左辺はよらないので,右辺の$r$についての
微分が$0$であることを示した後,$r \to 0$の極限を取る.

\begin{ans}
    示すべき式の右辺を$\phi(r)$とおく.
    境界条件と$-\triangle u = f$から
    \begin{equation}
        \phi (r) = \frac{1}{n\alpha(n)r^{n-1}} \int_{\partial B(0,r)} u(x)dS(x) 
        + \frac{1}{n(n-2)\alpha(n)} 
        \int_{B(0,r)} \paren{ \frac{1}{r^{n-2}} - \frac{1}{\abs{x}^{n-2}} }\triangle udx
    \end{equation}
    となる.
    \begin{equation}
        \int_{B(0,r)} \frac{\triangle u}{\abs{x}^{n-2}} dx 
        = \int_{0}^{r} \int_{\partial B(0,s)} \frac{\triangle u}{\abs{x}^{n-2}} dS(x) ds
        = \int_{0}^{r} \frac{1}{s^{n-2}} \int_{\partial B(0,s)} \triangle u(x) dS(x) ds
    \end{equation}
    だから
    \begin{equation}
        \frac{d}{dr} \int_{B(0,r)} \frac{\triangle u}{\abs{x}^{n-2}} dx 
        = \frac{1}{r^{n-2}} \int_{\partial B(0,r)} \triangle u(x) dS(x)
    \end{equation}
    となることに注意すれば,
    \begin{align}
        \phi'(r) &=
        \frac{1}{n\alpha(n)r^{n-1}} \int_{B(0,r)} \triangle u(x)dx
        + \frac{1}{n(n-2)\alpha(n)} \frac{1}{r^{n-2}}
         \int_{\partial B(0,r)} \triangle u(x) dS(x) \\
        &\quad + \frac{1}{n(n-2)\alpha(n)} \frac{-(n-2)}{r^{n-1}} 
        \int_{B(0,r)} \triangle u(x)dx 
        - \frac{1}{n(n-2)\alpha(n)}\frac{1}{r^{n-2}} \int_{\partial B(0,r)} \triangle u(x) dS(x)\\
        &=0
    \end{align}
    となる.したがって$\phi(r)$は定数である.平均値の公式の導出と同様に
    \begin{equation}
        \lim_{r\to 0} \frac{1}{n\alpha(n)r^{n-1}} \int_{\partial B(0,r)} u(x)dS(x) = u(0)
    \end{equation}
    となる.
    \begin{equation}
        \lim_{r\to 0} \frac{1}{r^{n-2}} \int_{B(0,r)} \triangle u dx 
        = \lim_{r \to 0} -\frac{\alpha(n)r^{2}}{\alpha(n)r^{n}} \int_{B(0,r)} \triangle f(x) dx 
        = 0
    \end{equation}
    となることも同様にして分かる.
    \begin{equation}
        \int_{B(0,r)} \frac{\triangle u}{\abs{x}^{n-2}} dx 
        = \int_{0}^{r} \frac{1}{s^{n-2}} \int_{\partial B(0,s)} \triangle u(x) dS(x) ds
        = \int_{0}^{r} \frac{-n\alpha(n)s}{n\alpha(n)s^{n-1}} 
        \int_{\partial B(0,s)} f(x) dS(x) ds
    \end{equation}
    であり,$r$が十分小さければ
    $\frac{1}{n\alpha(n)s^{n-1}} 
    \int_{\partial B(0,s)} f(x) dS(x)$は$f(0)$に近づく.
    したがって
    \begin{equation}
        \lim_{r\to 0} \int_{B(0,r)} \frac{\triangle u}{\abs{x}^{n-2}} dx 
        = \lim_{r\to 0} \int_{0}^{r} -n\alpha(n)s f(0) ds = 0
    \end{equation}
    となる.以上により$\lim_{r\to 0} \phi(r) = u(0)$となるから,
    $\phi(r) = u(0)$である.よって示された.
\end{ans}


\subsection{方針と解答}
最大値を達成する点ではヘッシアンが半負定値であることを用いる.
ヒントの通りにおいてみる.

\begin{ans}
    $\max_{\bar{U}} u \leq \max_{\partial U} u$を示せば十分である.
    $\varepsilon>0$に対して
    $u_{\varepsilon}(x) \coloneqq u(x) + \varepsilon \abs{x}^{2}$とおくと,
    $\mathrm{Hess} u_{\varepsilon} = \mathrm{Hess} u + 2\varepsilon I$が成り立つ.
    $u_{\varepsilon}$が$x \in U$で最大値を達成すると仮定すると,
    $\mathrm{Hess} u_{\varepsilon}(x)$が半負定値であることから,
    $i$成分が$1$の単位ベクトル$e_{i}$に対して
    $\iprod{\mathrm{Hess} u_{\varepsilon}(x) e_{i}, e_{i}} 
    = \frac{\partial^{2}u}{\partial x_{i}^{2}}(x) + 2\varepsilon \leq 0$
    となるから,$\triangle u (x) + 2n\varepsilon \leq 0$となる.$\triangle u=0$
    だから,これは矛盾であり,$u_{\varepsilon}$は$U$で最大値を達成することはない.したがって
    $\max_{\bar{U}} u_{\varepsilon} \leq \max_{\partial U} u_{\varepsilon}$が成り立つ.
    $M \coloneqq \max_{\bar{U}} \abs{x}^{2}$とおき,$\varepsilon_{n} 
    \coloneqq 1/nM$とする.
    $\max_{\bar{U}} u_{\varepsilon_{n}} \leq \max_{\partial U} u_{\varepsilon_{n}}$
    から
    $\max_{\bar{U}} u - 1/n \leq \max_{\partial U}u + 1/n$が成り立つ.
    $n \to \infty$として,示された.
\end{ans}

\subsection{方針と解答}
問題文のいう通りに計算するだけ.劣調和は大切な性質ではある.
$U$は連結開集合として解答する.

\begin{ans}
    (a).$\phi(r) \coloneqq \intbar_{B(x,r)} v dy$とおき,微分する.

    (b).$x \in U$で最大値を取ったと仮定すると,ある$r>0$が存在して
    $B(x,r)$上$v(y) \leq v(x)$となる.このとき$(a)$から
    $v(x) \leq \intbar_{B(x,r)} v(y)dy \leq v(x)$となって,
    $B(x,r)$上$v$は定数.$U$が連結だから$U$全体で定数となって境界上でも定数.
    結局$\max_{\bar{U}} v = \max_{\partial U} v$となる.

    (c).$\frac{\partial v}{\partial x_{i}} = 
    \frac{\partial}{\partial x_{i}} \phi(u(x))
    = \phi'(u(x)) \frac{\partial u}{\partial x_{i}}$および
    $\frac{\partial^{2} v}{\partial x_{i}^{2}} 
    = \phi''(u(x)) \paren{\frac{\partial u}{\partial x_{i}}}^{2}
    + \phi'(u(x)) \frac{\partial^{2} u}{\partial x_{i}^{2}}$
    から$\triangle v = \sum_{i=1}^{n} \phi''(u(x)) 
    \paren{\frac{\partial u}{\partial x_{i}}}^{2} \geq 0$となる.
    最後の不等式で$\phi$の凸性を用いた.

    (d).計算すると$\triangle v(x) = \sum_{i,j=1}^{n} 
    2\paren{\frac{\partial^{2}u}{\partial x_{i} \partial x_{j}}}^{2} \geq 0$を得る.
\end{ans}
\end{document}