\documentclass[dvipdfmx,autodetect-engine]{jsarticle}


\usepackage[utf8]{inputenc}
\usepackage{amsmath}
\usepackage{amsthm}
\usepackage{amssymb}
\usepackage{mathtools}

\usepackage[dvipdfmx]{graphicx}

\newtheorem{theorem}{Theorem}[section]
\newtheorem{corollary}{Corollary}[theorem]
\newtheorem{lemma}[theorem]{Lemma}
\newtheorem{example}{Example}[section]
\newtheorem*{ans}{解答}

\theoremstyle{remark}
\newtheorem*{remark}{Remark}

\theoremstyle{definition}
\newtheorem{definition}{Definition}[section]
\mathtoolsset{showonlyrefs}
\allowdisplaybreaks

\renewcommand{\labelenumi}{(\arabic{enumi})}
\renewcommand{\labelenumii}{(\alph{enumii}}
\newcommand{\R}{\mathbb{R}}
\newcommand{\N}{\mathbb{N}}
\newcommand{\C}{\mathbb{C}}
\newcommand{\Z}{\mathbb{Z}}
\newcommand{\diver}{\mathrm{div} \,}
\newcommand{\rot}{\mathrm{rot} \,}
\newcommand{\abs}[1]{\left\lvert#1\right\rvert}
\newcommand{\norm}[1]{\left\lVert#1\right\rVert}
\newcommand{\setmid}{\mathrel{} \middle| \mathrel{}}
\newcommand{\paren}[1]{\left( #1 \right)}
\newcommand{\iprod}[1]{\left\langle #1 \right\rangle}


\begin{document}

\title{Evans 演習問題解答}
\author{@litharge3141}
\date{\today}
\maketitle

\abstract
Evans, Partial Differential Equationsの演習問題の解答.
問題は載せません.

\section{1章の問題}
略
\section{2章の問題}
\subsection{方針と解答}
$cu$がなければ解けるので$cu$を非斉次項だと思って定数変化法を用いる.
\begin{ans}
    $v(t,x) = g(x-bt)$とおく.
    $v_{t} = - b \cdot Dv$が満たされることに注意する.
    $u(t,x) = \varphi(t) v(t,x)$とおいて方程式に代入すると
    \begin{align}
        &\varphi'(t) v(t,x) + \varphi(t) \partial_{t} v(t,x) + b\cdot Du(t,x) + c
        \varphi(t)v(t,x)\\
        &=\varphi'(t) v(t,x)+ c\varphi(t)v(t,x) = 0
    \end{align}
    から$\varphi'(t) = -c \varphi(t)$を得る.
    よって$\varphi(t) = Ae^{-ct}$となり,初期条件と合わせて
    $u(t,x) = e^{-ct}g(x-bt)$を得る.
\end{ans}

\subsection{方針と解答}
公式を導くつもりで成分計算をする.$O$は書きづらいので$U$とかにしてほしかったです.
\begin{ans}
    $O$が直交行列であることから,任意の$1\leq i,j \leq n$に対して
    $\sum_{k=1}^{n} O_{ik} O'_{kj} = \sum_{k=1}^{n} O_{ik} O_{jk} = \delta_{ij}$
    が成り立つことに注意する.ここで$\delta_{ij}$は$i=j$のとき$1$でそれ以外は$0$として
    定める(単位行列の$i,j$成分).
    $u$が調和関数であると仮定する.
    $1\leq s\leq n$に対して,
    \begin{equation}
        \frac{\partial}{\partial x_{s}} u(Ox)
        = \sum_{i=1}^{n} u_{x_{i}}(Ox) \frac{\partial (Ox)_{i}}{\partial x_{s}}
    \end{equation}
    となる.$(Ox)_{i} = \sum_{j=1}^{n} O_{ij}x_{j}$となるから
    $\frac{\partial (Ox)_{i}}{\partial x_{s}} = O_{is}$となる.
    したがって
    \begin{equation}
        \frac{\partial^{2}}{\partial x_{s}^{2}} u(Ox)
        = \frac{\partial}{\partial x_{s}}
        \sum_{i=1}^{n} u_{x_{i}}(Ox) O_{is} 
        = \sum_{j=1}^{n} \sum_{i=1}^{n} u_{x_{i}x_{j}}(Ox) O_{is} O_{js}
    \end{equation}
    となるから,
    \begin{align}
        &\triangle u(Ox) = \sum_{s=1}^{n}  \frac{\partial^{2}}{\partial x_{s}^{2}} u(Ox)
        = \sum_{j=1}^{n} \sum_{i=1}^{n} u_{x_{i}x_{j}}(Ox) \sum_{s=1}^{n}O_{is} O_{js}
        = \sum_{j=1}^{n} \sum_{i=1}^{n} u_{x_{i}x_{j}}(Ox) \delta_{ij} \\
        &= \sum_{i=1}^{n} u_{x_{i}x_{i}}(Ox) = 0\quad(\because u\text{は調和関数})
    \end{align}
    により結論を得た.
\end{ans}

\end{document}