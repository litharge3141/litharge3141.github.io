\documentclass[dvipdfmx,autodetect-engine]{jsarticle}


\usepackage[utf8]{inputenc}
\usepackage{amsmath}
\usepackage{amsthm}
\usepackage{amssymb}
\usepackage{mathtools}

\usepackage[dvipdfmx]{graphicx}

\newtheorem{theorem}{Theorem}[section]
\newtheorem{corollary}{Corollary}[theorem]
\newtheorem{lemma}[theorem]{Lemma}
\newtheorem{example}{Example}[section]

\theoremstyle{remark}
\newtheorem*{remark}{Remark}

\theoremstyle{definition}
\newtheorem{definition}{Definition}[section]
\mathtoolsset{showonlyrefs}

\renewcommand{\labelenumi}{(\arabic{enumi})}
\renewcommand{\labelenumii}{(\alph{enumii}}
\newcommand{\R}{\mathbb{R}}
\newcommand{\N}{\mathbb{N}}
\newcommand{\C}{\mathbb{C}}
\newcommand{\Z}{\mathbb{Z}}
\newcommand{\diver}{\mathrm{div} \,}
\newcommand{\rot}{\mathrm{rot} \,}
\newcommand{\abs}[1]{\left\lvert#1\right\rvert}
\newcommand{\norm}[1]{\left\lVert#1\right\rVert}


\begin{document}

\title{SDEの数値計算 強収束}
\author{@litharge3141}
\date{\today}
\maketitle

\abstract{}
SDEの数値計算の中で最も基本的なEuler-MaruyamaおよびMilsteinのスキームについて,
その厳密解への強収束と呼ばれる収束について述べる.
これは数値計算の各時間ステップについて,その厳密解からの分散が
刻み幅$h$を用いて上から評価できるというものである.

\section{}

\subsection{準備}
主定理を述べる前に必要な用語などについて述べる.
数値計算においては確率微分方程式の強解を近似して計算する.
\begin{definition}
    $(\Omega,\mathcal{F},P,(\mathcal{F}_{t})_{t \in [0,T]})$を
    フィルトレーション付き確率空間とし,$(B(t))_{t \in [0,T]}$を
    $m$次元$\mathcal{F}_t$-ブラウン運動とする.
    $1\leq i\leq n, 1\leq r \leq m$に対して,Borel可測な函数
    $a^{i},\sigma_{r}^{i} \colon [0,T] \times \R^{n} \to \R$
    が与えられているとする.このとき,確率過程$(X(t))_{t \in [0,T]}$が
    $x \in R^{n}$を出発点とする確率微分方程式
    \begin{align}
        dX(t) = a(t,X(t)) dt + \sum_{r=1}^{m} \sigma_{r}(t,X(t)) d B^{r}(t)
    \end{align}
    あるいは成分ごとに書いた
    \begin{align}
        dX^{i} (t) = a^{i}(t,X(t)) dt + \sum_{r=1}^{m} \sigma_{r}^{i}(t,X(t)) dB^{r}(t)
    \end{align}
    の強解であるとは,
    \begin{enumerate}
        \item $X(t)$は可測かつ$\mathcal{F}_t$-適合な連続確率過程である.
        \item 任意の$1\leq i\leq n$と$1\leq r\leq m$に対して,
        $\sigma_{r}^{i}(t,X(t)) \in \mathcal{L}^2(\mathcal{F}_t)$
        かつ$a(t,X(t)) \in L^1[0,T]$が満たされる.
        \item $X(t)$は確率積分方程式
        \begin{align}
            X(t) = x + \int_{0}^{t} a(s,X(s))ds + \int_{0}^{t} \sigma_{r}(s,X(s))dB^{r}(s)
        \end{align}
        を満たす.
    \end{enumerate}
    という条件を満たすことをいう.
\end{definition}


強解の存在と一意性については次の定理がよく知られている.
証明は省略する.


\begin{theorem}
    係数$a,\sigma_{r}$が以下を満たすと仮定する.
    \begin{enumerate}
        \item Lipshitz連続,すなわち,
        \begin{align}
            \exists K>0,\quad \forall t \in [0,T],\quad \forall x,y \in \R^{n},\quad
            \abs{a(t,x)-a(t,y)} + \sum_{r=1}^{m} \abs{\sigma_{r}(t,x) - (t,y)} 
            \leq K\abs{x-y}
        \end{align}
        を満たす.
        \item 1次増大条件,すなわち
        \begin{align}
            \exists K>0,\quad \forall t \in [0,T],\quad \forall x\in \R^{n},\quad
            \abs{a(t,x)} + \sum_{r=1}^{m} \abs{\sigma_{r}(t,x)} \leq K(1+\abs{x})
        \end{align}
        を満たす.
    \end{enumerate}
    このとき,確率微分方程式の強解$X(t)$で,各成分が$\mathcal{L}^2$に属するものが
    存在する.さらに
    $\tilde{X}(t)$も強解ならば,$P(\forall t \geq 0,X(t)=\tilde{X}(t))=1$
    が成り立つという意味で,解$X(t)$は一意である.
\end{theorem}


\end{document}