\documentclass[dvipdfmx,autodetect-engine]{jsarticle}


\usepackage[utf8]{inputenc}
\usepackage{amsmath}
\usepackage{amsthm}
\usepackage{amssymb}
\usepackage{mathtools}

\usepackage[dvipdfmx]{graphicx}

\newtheorem{theorem}{Theorem}[section]
\newtheorem{corollary}{Corollary}[theorem]
\newtheorem{lemma}[theorem]{Lemma}
\newtheorem{example}{Example}[section]
\newtheorem*{prob}{問題}
\newtheorem*{ans}{解答}

\theoremstyle{remark}
\newtheorem*{remark}{Remark}

\theoremstyle{definition}
\newtheorem{definition}{Definition}[section]
\mathtoolsset{showonlyrefs}
\allowdisplaybreaks

\renewcommand{\labelenumi}{(\arabic{enumi})}
\renewcommand{\labelenumii}{(\alph{enumii}}
\newcommand{\R}{\mathbb{R}}
\newcommand{\N}{\mathbb{N}}
\newcommand{\C}{\mathbb{C}}
\newcommand{\Z}{\mathbb{Z}}
\newcommand{\diver}{\mathrm{div} \,}
\newcommand{\rot}{\mathrm{rot} \,}
\newcommand{\abs}[1]{\left\lvert#1\right\rvert}
\newcommand{\norm}[1]{\left\lVert#1\right\rVert}
\newcommand{\setmid}{\mathrel{} \middle| \mathrel{}}
\newcommand{\paren}[1]{\left( #1 \right)}
\newcommand{\iprod}[1]{\left\langle #1 \right\rangle}


\begin{document}

\title{演習問題解答}
\author{@litharge3141}
\date{\today}
\maketitle

\abstract
Karatzas-Shreve, Brownian Motion and Stochastic Calculusの
ExerciseとProblemの解答.問題は載せません.

\section{Chapter1}
\begin{ans}[1.5 Problem]
    右連続性を利用して連続な時間を可算に落とす.

    任意の$t \geq 0$に対して$P(X_{t}=Y_{t})=1$が成立する.
    $[0,\infty)$の稠密な可算集合$(t_{m})_{m=1}^{\infty}$を取る.
    任意の$m \in \N$に対してある$P$-零集合$N_{m}$が存在して,
    $\omega \notin N_{m}$ならば$X_{t_{m}}(\omega) = Y_{t_{m}}(\omega)$が成立する.
    そこで$N = \bigcup_{m=1}^{\infty} N_{m}$とおくと,
    $N$は$P$-零集合で,$\omega \notin N$ならば任意の$m\in \N$に対して
    $X_{t_{m}}(\omega) = Y_{t_{m}}(\omega)$が成立する.
    すなわち,$P(\forall m\in \N,\, X_{t_{m}} = Y_{t_{m}})=1$となる.
    $X,Y$はほとんどいたるところ右連続だから,$N_{X},N_{Y}$という$P$-零集合を
    除いて右連続である.$N \cup N_{X} \cup N_{Y}$を改めて$N$とおく.
    $N$は$P$-零集合である.$\omega \notin N$とする.
    任意の$t \geq 0$に対して,$t$に右から収束する$(t_{m})_{m=1}^{\infty}$
    の部分列$(t_{m(k)})_{k=1}^{\infty}$が稠密性から存在する.
    任意の$k\in \N$に対して$X_{t_{m(k)}}(\omega) = Y_{t_{m(k)}}(\omega)$
    が成立することと,右連続性から$X_{t}(\omega) = Y_{t}(\omega)$となる.
    $t$は任意だったから,$P(\forall t\geq 0,\, X_{t}=Y_{t})=1$となる.
    以上により示された.
\end{ans}
右連続でなく左連続でもできそうな感じがするが,$t=0$のところの処理
(というか左連続の定義)が若干面倒であるように思われる.


\begin{ans}[1.7 Exercise]
    「極限が存在する」という事象を可算な操作で言い換えること,
    RCLLなら不連続な点は高々可算であることに注目する.

    $t_{0} \in (0,\infty)$が任意に与えられたとする.
    任意の$\omega \in \Omega$に対して$X(\omega)$はRCLLだから
    不連続点は高々可算で,$(0,t_{0})$上の
    不連続点の全体を$(t_{k})_{k=1}^{\infty}$とおける
    ($\omega$に依存することに注意).
    また,$X(\omega)$は右連続だから,$X(\omega)$が$(0,t_{0})$で左連続であることと
    $(0,t_{0})$で連続であることは同値.したがって
    \begin{equation}
        \omega \in A 
        \Leftrightarrow \forall k \in \N, 
        \lim_{s \to t_{k}-0} X_{s}(\omega) = X_{t_{k}}(\omega)
    \end{equation}
    が成り立つ.
    $\lim_{s \to t_{k}-0} X_{s}(\omega) = X_{t_{k}}(\omega)$が成り立つことは,
    任意の$m \in \N$に対してある$N \in \N$が存在して$n \geq N$ならば
    $\abs{X_{t_{k}-1/n}(\omega) - X_{t_{k}}(\omega)} < 1/m$が成立することと同値.
    後者は
    \begin{equation}
        \omega \in \bigcap_{m=1}^{\infty} \bigcup_{N=1}^{\infty} \bigcap_{n=N}^{\infty}
        \left\{ \abs{X_{t_{k}-\frac{1}{n}} - X_{t_{k}}} < \frac{1}{m} \right\}
    \end{equation}
    と書き直せる.$\{\abs{X_{t_{k}-1/n} - X_{t_{k}}} < 1/m\} \in \mathcal{F}_{t_{k}}^{X}$
    だから,$\bigcap_{m=1}^{\infty} \bigcup_{N=1}^{\infty} \bigcap_{n=N}^{\infty}
    \{\abs{X_{t_{k}-1/n} - X_{t_{k}}} < 1/m\} \in \mathcal{F}_{t_{k}}^{X}$である.
    任意の$k\in \N$に対して$\mathcal{F}_{t_{k}}^{X} \subset \mathcal{F}_{t_{0}}^{X}$
    となる.したがって
    \begin{equation}
        A = \bigcap_{k=1}^{\infty} \bigcap_{m=1}^{\infty} \bigcup_{N=1}^{\infty} \bigcap_{n=N}^{\infty}
    \{\abs{X_{t_{k}-1/n} - X_{t_{k}}} < 1/m\} \in \mathcal{F}_{t_{0}}^{X}
    \end{equation}
    となり,示された.
\end{ans}

\begin{ans}[1.16 Problem]
    $X$が直積可測だから,
    可測集合の$X$による引き戻しはほとんど直積集合の形で書けることを利用する.

    $E$を$X$の行先の任意の可測集合とする.$X_{T}^{-1}(E) \in \mathcal{F}$を
    示せばよい.
    \begin{equation}
        X_{T}^{-1}(E) = \left\{\omega \setmid X_{T(\omega)}(\omega) \in E\right\}
        = \left\{ \omega \setmid (T(\omega),\omega) \in X^{-1}(E)\right\}
    \end{equation}
    となるから,任意の$F\in \mathcal{B}([0,\infty))\otimes \mathcal{F}$に
    対して$\left\{ \omega \setmid (T(\omega),\omega) \in F\right\} \in \mathcal{F}$
    が成り立つことを示せば$X$が可測であることから$X_{T}^{-1}(E) \in \mathcal{F}$が従う.
    \begin{equation}
        \mathcal{D} \coloneqq 
        \left\{ F \subset [0,\infty) \times \Omega
        \setmid \{\omega \setmid (T(\omega),\omega) \in F\}\in \mathcal{F}\right\}
    \end{equation}
    とおく.$\mathcal{D}$が$\sigma$-代数であることはやるだけなので省略する.
    $F = E^{t} \times E^{\Omega}, E^{t} \in \mathcal{B}([0,\infty)),
    E^{\Omega} \in \mathcal{F}$とすると$T$は可測だから
    $\left\{\omega \setmid (T(\omega),\omega) \in F\right\}= 
    E^{\Omega} \cap T^{-1}(E^{t}) \in \mathcal{F}$となり,$F \in \mathcal{D}$
    となる.したがって直積$\sigma$-代数の定義から
    $\mathcal{B}([0,\infty))\otimes \mathcal{F} \subset \mathcal{D}$となり,
    示された.

\end{ans}


\end{document}