\documentclass[dvipdfmx,autodetect-engine]{jsarticle}


\usepackage[utf8]{inputenc}
\usepackage{amsmath}
\usepackage{amsthm}
\usepackage{amssymb}
\usepackage{mathtools}

\usepackage[dvipdfmx]{graphicx}

\newtheorem{theorem}{Theorem}[section]
\newtheorem{corollary}{Corollary}[theorem]
\newtheorem{lemma}[theorem]{Lemma}
\newtheorem{example}{Example}[section]

\theoremstyle{remark}
\newtheorem*{remark}{Remark}

\theoremstyle{definition}
\newtheorem{definition}{Definition}[section]
\mathtoolsset{showonlyrefs}

\renewcommand{\labelenumi}{(\arabic{enumi})}
\renewcommand{\labelenumii}{(\alph{enumii}}
\newcommand{\R}{\mathbb{R}}
\newcommand{\N}{\mathbb{N}}
\newcommand{\C}{\mathbb{C}}
\newcommand{\Z}{\mathbb{Z}}
\newcommand{\diver}{\mathrm{div} \,}
\newcommand{\rot}{\mathrm{rot} \,}
\newcommand{\abs}[1]{\left\lvert#1\right\rvert}
\newcommand{\norm}[1]{\left\lVert#1\right\rVert}
\newcommand{\setmid}{\mathrel{} \middle| \mathrel{}}
\newcommand{\paren}[1]{\left( #1 \right)}
\newcommand{\iprod}[1]{\left\langle #1 \right\rangle}


\begin{document}

\title{SPDEの数値計算}
\author{@litharge3141}
\date{\today}
\maketitle

\section{Introduction}
空間一次元の熱方程式を例にして,SPDEの導入をする.
あらすじとしては,Hilbert空間の完全正規直交系との内積をとって
係数についてのSDEに帰着できるということであるが,
無限次元で考えるので収束の問題が常について回ることに注意する.
まずノイズの定義をする.
今後,確率空間$(\Omega,\mathcal{F},P)$の
フィルトレーションは右連続かつ$P$-零集合をすべて含むとする.


\begin{definition}[柱状Brown運動]
$H$を実Hilbert空間,$T>0$として
$(\Omega,\mathcal{F},P,(\mathcal{F}_{t})_{t\in [0,T]})$を
フィルトレーション付き確率空間とする.
$\norm{\cdot}_{H}$で$H$のノルムを表すことにする.
$W \colon H  \times [0,T] \times \Omega \to \R$が$H$上の
柱状Brown運動であるとは,
\begin{itemize}
    \item 任意の$0$でない$\psi \in H$に対して$W(\psi,\cdot,\cdot) / \norm{\psi}_{H}
    \colon [0,T]\times\Omega  \to \R$は実$\mathcal{F}_{t}$-Brown運動である.
    \item 任意の$\alpha,\beta \in \R$と任意の$\varphi,\psi \in H$に対して
    \begin{equation}
        P\paren{\omega \setmid \forall t \in [0,T],\,
        W(\alpha\psi + \beta\varphi,t,\omega) = 
    \alpha W(\psi,t,\omega) + \beta W(\varphi, t, \omega)} =1
    \end{equation}
    が成り立つ.
\end{itemize}
の二条件が成り立つことをいう.柱状Brown運動$W$に対して,
$W(\psi,t,\cdot)$を単に$W_{t}(\psi)$と書くこともある.
\end{definition}


\begin{theorem}
    $(H,\iprod{\cdot,\cdot}_{H})$を可分無限次元実Hilbert空間,
    $(e_{k})_{k=1}^{\infty}$を$H$の可算な完全正規直交系とする.
    $(\Omega,\mathcal{F},P,(\mathcal{F}_{t})_{t\in [0,T]})$を
    フィルトレーション付き確率空間とし,
    $(B_{t}^{n})_{k=1}^{\infty}$を独立な$\mathcal{F}_{t}$-ブラウン運動の族とする.
    このとき,$W \colon H \times [0,T]\times \Omega  \to \R$を
    $W(\psi,t,\omega) \coloneqq \sum_{k=1}^{\infty} B_{t}^{k}(\omega) \iprod{
    \psi, e_{k}}_{H}$によって定めると$W$はwell-definedで,柱状Brown運動になる.
\end{theorem}


\begin{proof}
    $n\in\N$に対して
    $W^{n}_{t} (\psi) \coloneqq W^{n}(\psi,t,\omega) \coloneqq \sum_{k=1}^{n} B_{t}^{k}(\omega) \iprod{
    \psi, e_{k}}_{H}$とおく.$(W^{n}(\psi,\cdot,\cdot))_{n=1}^{\infty}$
    が$M_{T}^{2}$のコーシー列であることを示す.
    \begin{align}
        &E\left[\paren{W_{T}^{n}(\psi) - W_{T}^{m}(\psi)}^{2}\right]
        = E\left[\paren{\sum_{k=m+1}^{n} B_{T}^{k} \iprod{\psi,e_{k}}_{H} }^{2}\right] \\
        &= \sum_{k=m+1}^{n} E\left[\paren{B_{T}^{k}}^{2}\right] \iprod{\psi,e_{k}}_{H}^{2} 
        = T\sum_{k=m+1}^{n} \iprod{\psi,e_{k}}_{H}^{2}
    \end{align}
    より,$\sum_{k=1}^{\infty} \iprod{\psi,e_{k}}_{H}^{2} = \norm{\psi}_{H}^{2}$
    であることから従う.したがって$W$はwell-definedで,$W_{t}(\psi) \in M_{T}^{2}$である.
    次に,$W(\psi)/\norm{\psi}_{H}$が実$\mathcal{F}_{t}$-Brown運動であることを示す.
    Levyの定理から$\iprod{W(\psi)/\norm{\psi}_{H}}_{t} = t$を示せば十分で,
    特に$\paren{W_{t}(\psi) / \norm{\psi}_{H}}^{2} - t$がマルチンゲールであることを
    証明すれば十分である.$0\leq s<t\leq T$が任意に与えられたとする.
    $\Phi$を$\mathcal{F}_{s}$-可測かつ有界な関数とすると,
    \begin{align}
        &E\left[ \paren{W_{t}(\psi)^{2} - W_{s}(\psi)^{2}}\Phi \right]\\
        &= \lim_{n\to\infty} E\left[\paren{W_{t}^{n}(\psi)^{2} - W_{s}^{n}(\psi)^{2}}\Phi\right]\\
        &= \lim_{n\to\infty} 
            E\left[
                \paren{
            \paren{
                \sum_{k=1}^{n} B_{t}^{k} \iprod{\psi,e_{k}}_{H} 
            }^{2}
             -
            \paren{
                 \sum_{k=1}^{n} B_{s}^{k}\iprod{\psi,e_{k}}_{H} 
            }^{2}
                }\Phi
            \right]\\
        &= \lim_{n\to\infty} 
        E\left[
            \paren{
        \paren{
            \sum_{k=1}^{n} (B_{t}^{k}-B_{s}^{k}) \iprod{\psi,e_{k}}_{H} 
        }^{2}
         +
        \paren{
             2\sum_{k=1}^{n}\sum_{l=1}^{n} (B_{t}^{k} -B_{s}^{k})B_{s}^{l}
             \iprod{\psi,e_{k}}_{H} \iprod{\psi,e_{l}}_{H}
        }
            }\Phi
        \right]\\
        &=\lim_{n\to\infty} 
        \sum_{k=1}^{n} E\left[
        \paren{
            B_{t}^{k}-B_{s}^{k}
        }^{2} 
        \right]
        E\left[ \Phi \right]
        \iprod{\psi,e_{k}}_{H}^{2} 
         +
        2\sum_{k=1}^{n}\sum_{l=1}^{n} \iprod{\psi,e_{k}}_{H} \iprod{\psi,e_{l}}_{H}
        E\left[
             B_{t}^{k} -B_{s}^{k}
        \right]
        E\left[
            B_{s}^{l}\Phi
        \right]\\
        &= (t-s) E\left[\Phi \right] \norm{\psi}_{H}^{2}
    \end{align}
    が成り立つ.特に$\Phi$として$\mathcal{F}_{s}$-可測な集合の定義関数をとれば,
    $\paren{W_{t}(\psi) / \norm{\psi}_{H}}^{2} - t$がマルチンゲールであることが従う.
    よって示された.最後に線形性を証明する.
    $\alpha,\beta\in \R$および$\psi,\varphi \in H$が任意に与えられたとする.
    $t \in [0,T]$の稠密な可算集合$(t_{m})_{m=1}^{\infty}$が与えられたとする.
    任意の$n \in \N$と任意の$\omega \in \Omega$に対して
    \begin{equation}
        W_{t_{m}}^{n}(\alpha \psi + \beta \varphi) 
        = \alpha W_{t_{m}}^{n}(\psi) + \beta W_{t_{m}}^{n}(\varphi)
    \end{equation}
    が成立する.ここで$W_{t_{m}}^{n} (\cdot)$は$W_{t_{m}}(\cdot)$に
    $L^{2}(\Omega)$で収束するので,必要なら部分列を取ることで
    ある$P(E_{m})=0$となる$E_{m}\in \mathcal{F}$が存在して,
    $\omega \notin E_{m}$ならば
    \begin{equation}
        W_{t_{m}}(\alpha \psi + \beta \varphi) 
        = \alpha W_{t_{m}}(\psi) + \beta W_{t_{m}}(\varphi)
    \end{equation}
    が成立する.$E = \bigcup_{m=1}^{\infty} E_{m}$とすると
    $P(E) = 0$であり,$\omega \notin E$ならば任意の$m \in \N$に対して
    \begin{equation}
        W_{t_{m}}(\alpha \psi + \beta \varphi) 
        = \alpha W_{t_{m}}(\psi) + \beta W_{t_{m}}(\varphi)
    \end{equation}
    が成立する.$W$は$t$について連続で,$(t_{m})_{m=1}^{\infty}$が$[0,T]$で
    稠密であることから,
    $\omega \notin E$ならば
    任意の$t \in [0,T]$に対して
    \begin{equation}
        W_{t}(\alpha \psi + \beta \varphi) 
        = \alpha W_{t}(\psi) + \beta W_{t}(\varphi)
    \end{equation}
    が成立する.したがって,線形性も満たされ,証明が終わった.
\end{proof}

$\sum_{k=1}^{\infty} B_{t}^{k} \iprod{\cdot, e_{k}}_{H}$は
柱状Brown運動であることが分かった.もし$\sum_{k=1}^{\infty} 
B_{t}^{k}(\omega)e_{k}$が$H$のノルムでほとんどいたるところ収束すれば,
$\sum_{k=1}^{\infty} B_{t}^{k} \iprod{\cdot, e_{k}}_{H}
= \iprod{\cdot, \sum_{k=1}^{\infty} B_{t}^{k}(\omega)e_{k}}_{H}$
がほとんどいたるところ成立するから$H$に値を取る確率過程
$\sum_{k=1}^{\infty} B_{t}^{k}(\omega)e_{k}$と
同一視できる.残念ながら$H$の元としては収束しないので,
このような見方は正当化されない.
$H$の元と同一視できるような場合として,色付きノイズ
と呼ばれるノイズを考える.まず,
Hilbert空間に値を取るノイズを考える.

\begin{definition}
    $(H,\iprod{\cdot,\cdot})$を可分実Hilbert空間とし,
    $T>0$として
    $(\Omega,\mathcal{F},P)$を
    フィルトレーション付き確率空間とする.
    $M\colon [0,T] \times \Omega \to H$が確率過程であるとは,
    $\norm{M_{t}(\omega)}_{H}$が実数値の確率過程となることをいう.
    $H$が可分だから,任意の$\varphi \in H$に対して
    $\iprod{\varphi, M_{t}(\omega)}_{H}$が実数値確率過程であるとしてもよい.
\end{definition}


このように可測性を定義しておけば$M$のモーメントをBochner積分やDunford積分で
定めることができる.特に今後必要な$L^2$-連続マルチンゲールを定義する.


\begin{definition}
    $(H,\iprod{\cdot,\cdot})$を可分実Hilbert空間とし,
    $T>0$として
    $(\Omega,\mathcal{F},P,(\mathcal{F}_{t})_{t\in [0,T]})$を
    フィルトレーション付き確率空間とする.
    確率過程$M\colon [0,T] \times \Omega \to H$が$L^2$-連続マルチンゲールとは,
    任意の$\varphi \in H$に対して$\iprod{\varphi,M}_{H} \in M_{T}^{2}(\mathcal{F}_{t})$
    となることをいう.このとき,$M \in M_{T}^{2}(H)$と書く.
\end{definition}


今回はDunford積分で定義したが,Bochner積分で定義しても同じである.
色付きノイズの定義に入る.
可分Hilbert空間$H$上のトレースクラス作用素の全体を$C_{1}(H)$
とかき,Hilbert-Schmidtクラス作用素の全体を$C_{2}(H)$とかく.
$P \in C_{2}(H)$を取る.$Q = P^{*}P$として$Q$を定めると
$Q \in C_{1}(H)$であることに注意する.特に
$(e_{k})_{k=1}^{\infty}$を$H$の完全正規直交系とすると
$\sum_{k=1}^{\infty} \iprod{Qe_{k},e_{k}}_{H}$は$(e_{k})_{k=1}^{\infty}$
の取り方によらずに一定値に収束するので,これを$TrQ$と名づけるのであった.
都合のいい$(e_{k})_{k=1}^{\infty}$を取って
$P = \sum_{k=1}^{\infty} \sqrt{s_{k}} e_{k} \otimes e_{k}$および
$Q = \sum_{k=1}^{\infty} s_{k} e_{k} \otimes e_{k}$としてSchmidt展開すると,
$TrQ = \sum_{k=1}^{\infty} s_{k}  <\infty$となる.
柱状Brown運動$W = \sum_{k=1}^{\infty} B_{t}^{k} e_{k}$は
$H$に値をとる確率過程としては意味を持たない.しかし形式的に$P$を作用させて
\begin{align}
    PW &= \sum_{k=1}^{\infty} \sqrt{s_{k}} e_{k} \otimes e_{k} W\\
        &= \sum_{k=1}^{\infty} \sqrt{s_{k}} \iprod{\sum_{l=1}^{\infty} B_{t}^{l} e_{l},e_{k}}_{H} e_{k}\\
        &= \sum_{k=1}^{\infty} \sqrt{s_{k}} B_{t}^{k} e_{k}
\end{align}
として計算すると,一番最後は
$E[\norm{\sum_{k=1}^{\infty} \sqrt{s_{k}} B_{t}^{k} e_{k}}_{H}^{2}] = 
\sum_{k=1}^{\infty} s_{k} E[\paren{B_{t}^{k}}^{2}] = tTrQ$
となってほとんど至るところ$H$で収束し,$H$に値をとる確率過程として意味をもつ.
これを踏まえて,次のように定義する.

\begin{definition}
    $(H,\iprod{\cdot,\cdot}_{H})$を可分無限次元実Hilbert空間,
    $(e_{k})_{k=1}^{\infty}$を$H$の可算な完全正規直交系とする.
    $(\Omega,\mathcal{F},P,(\mathcal{F}_{t})_{t\in [0,T]})$を
    フィルトレーション付き確率空間とし,
    $(B_{t}^{n})_{k=1}^{\infty}$を独立な$\mathcal{F}_{t}$-ブラウン運動の族とする.
    $(s_{k})_{k=1}^{\infty}$を$\sum_{k=1}^{\infty} s_{k} < \infty$となる非負実数
    単調減少列とするとき,
    $Q$-Brown運動$W^{Q}\colon [0,T] \times \Omega \to H$を
    \begin{equation}
        W^{Q} \coloneqq \sum_{k=1}^{\infty} \sqrt{s_{k}} B_{t}^{k} e_{k}
    \end{equation}
    によって定義する.$Q$-Brown運動$W^{Q}$に対して
    $Q \coloneqq \sum_{k=1}^{\infty} s_{k} e_{k} \otimes e_{k}$と定めると
    $Q \in C_{1}(H)$であり,$Q$を共分散という.
\end{definition}


特に$H = L^{2}(X,\mu)$として$\sigma$-有限な測度空間上の$L^2$空間となるときは,
$Q$が定めるHilbert-Schmidt型積分作用素の積分核と同一視して
$Q(x,y) = \sum_{k=1}^{\infty} s_{k} e_{k}(x)e_{k}(y)$
のように書くこともある.先に$Q\in C_{1}(H)$を定めてから$W^{Q}$を定める方がスマートだが,
$Q$のSchmidt展開の仕方によらないことを示すのが大変そうなので,ここでは上のように定義した.


\begin{example}[熱方程式]
    $H = L^{2}[0,2\pi]$とする.
    以下では解の意味にこだわらず,形式的に計算して解の性質を予想する.
    \begin{align}
        &u_{t} = u_{xx} + W^{Q},\quad (t,x) \in (0,\infty)\times (0,2\pi)\\
        &u(0,x) = u_{0}(x),\quad u(t,0)=u(t,2\pi)
    \end{align}
    を考える.ここで$W^{Q}$は$Q$-Brown運動であり,
    $H$の完全正規直交系$e_{k} = \sqrt{1/\pi} \sin{\paren{kx/2}}$と
    $\sum_{k=1}^{\infty} q_{k} <\infty$となる減少列$(q_{k})_{k=1}^{\infty}$
    を用いて$W^{Q} = \sum_{k=1}^{\infty} \sqrt{q_{k}} B_{t}^{k} e_{k}$によって与えられる.
    $(e_{k})_{k=1}^{\infty}$は$\lambda_{k} \coloneqq k^{2}/4$とすると
    次を満たすことに注意する.
    \begin{equation}
        -\partial_{x}^{2} e_{k} = \lambda_{k} e_{k},\quad e_{k}(0) = e_{k}(2\pi) = 0,\quad
        k=1,2,\ldots
    \end{equation}
    $u=\sum_{k=1}^{\infty} u_{k}(t) e_{k}$の形で解を求める.
    $W^{Q}$と$u$の展開をもとの式に代入し,$e_{i}$との内積をとると
    \begin{align}
        &du_{i}(t) = -\lambda_{i} u_{i}(t)dt + \sqrt{q_{i}} dB_{t}^{i}\\
        &u_{i}(0) = \iprod{u_{0},e_{i}}_{H}
    \end{align}
    という確率微分方程式を得る.$i=1,2,\ldots$に対してこれらを解けばよい.
    この確率微分方程式自体は簡単にとけて
    \begin{align}
        u_{i}(t) = u_{i}(0) e^{-\lambda_{i}t} + 
        \sqrt{q_{i}} \int_{0}^{t} e^{-\lambda_{i}(t-s)}dB_{s}^{i}
    \end{align}
    として求まるから,
    \begin{equation}
        u(t,x) = \sum_{k=1}^{\infty} 
        \left[
        u_{k}(0) e^{-\lambda_{k}t} + 
        \sqrt{q_{k}} \int_{0}^{t} e^{-\lambda_{k}(t-s)}dB_{s}^{k}
        \right]e_{k}
    \end{equation}
    という形で解が求まる.
    解の性質を調べる.簡単のため$u_{0}=0$とする.あるいは平均からのずれが見たいと思ってもよい.
    このときは
    \begin{equation}\label{heat_Ito_calculus}
        u(t,x) = \sum_{k=1}^{\infty} 
        \left[
        \sqrt{q_{k}} \int_{0}^{t} e^{-\lambda_{k}(t-s)}dB_{s}^{k}
        \right]e_{k}
    \end{equation}
    となる.
    \begin{align}
        &E[\norm{u(t,x)}_{H}^{2}] = \sum_{k=1}^{\infty} E\left[
            q_{k} \paren{
                \int_{0}^{t} e^{-\lambda_{k}(t-s)} dB_{s}^{k}
            }^{2}
        \right]\\
        &= \sum_{k=1}^{\infty} q_{k} \int_{0}^{t} e^{-2\lambda_{k}(t-s)}ds
        \quad (\because \text{伊藤積分の等長性})\\
        &= \sum_{k=1}^{\infty} q_{k} \frac{1 - e^{-2\lambda_{k}t}}{2\lambda_{k}}
        \leq \sum_{k=1}^{\infty}  \frac{q_{k}}{2\lambda_{k}}
    \end{align}
    として$u(t,x)$のノルムの評価ができる.$\lambda_{k}$は$k^2$のオーダーで,
    $q_{k}$は総和が収束するから,$\sum_{k=1}^{\infty}  q_{k} / 2\lambda_{k}$は有限である.
    特に$q_{k}=1$の場合でも収束するから,表示$\eqref{heat_Ito_calculus}$自体は
    $q_{k}=1$の場合,すなわちノイズが柱状Brown運動の場合でも意味を持つことが期待できる.
    $u(t,x)$の$x$についての導関数を考える.一階の導関数は,$e_{k}$の微分から$k$倍
    が出てくるので,上の評価と同様にして,定数$M>0$を用いて形式的に
    \begin{equation}
        E[\norm{u_{x}(t,x)}_{H}^{2}] \leq M \sum_{k=1}^{\infty} \frac{k^{2}q_{k}}{\lambda_{k}}
    \end{equation}
    となる.$\sum_{k=1}q_{k}$が収束していないと右辺が有限ではないから,
    柱状ブラウン運動のときは導関数は通常の関数としての意味を持たないと考えられる.
    二階の導関数については$-\partial_{x}^{2} e_{k} = \lambda_{k} e_{k}$より
    \begin{equation}
        E[\norm{u_{xx}(t,x)}_{H}^{2}] \leq
         M \sum_{k=1}^{\infty} \frac{\lambda_{k}^{2}q_{k}}{\lambda_{k}}
         = M\sum_{k=1}^{\infty} \lambda_{k} q_{k}
    \end{equation}
    となる.今度は$q_{k}$が$p>3$に対して$k^{-p}$のオーダーでもないと右辺が有限ではない.
    したがって,柱状Brown運動の場合でも解の表示は意味を持ちそうであること,
    導関数は意味を持ちそうにないこと,$Q$-Brown運動の場合でも二階の導関数は
    意味を持つとは限らないことが予想される.
    数値計算をするにあたっては,導関数が意味を持たないことが話を難しくする.
    (以下は6/30時点での素人の考察なので真に受けないように)
    ノイズがブラウン運動の有限和でかけているような場合(退化しているという)には
    このような問題は生じない可能性があるし,もっと言えばランダムなPDEの
    計算も盛んに行われている.
    無限和でないと説明できないような解の性質を出さないとSPDEをモデルとして
    採用する理由が薄いような気がしないでもない(一般化としての価値はあると思うが)し,
    そしてそれは必ずあるはずである.
    今はわからなくても今後そういう性質を理解するのが目標の一つ.
\end{example}

\end{document}